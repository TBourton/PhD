% Copyright 2004 by Till Tantau <tantau@users.sourceforge.net>.
%
% In principle, this file can be redistributed and/or modified under
% the terms of the GNU Public License, version 2.
%
% However, this file is supposed to be a template to be modified
% for your own needs. For this reason, if you use this file as a
% template and not specifically distribute it as part of a another
% package/program, I grant the extra permission to freely copy and
% modify this file as you see fit and even to delete this copyright
% notice. 

\documentclass{beamer}
\DeclareMathOperator{\tr}{tr}
\DeclareMathOperator{\Tr}{Tr}
\DeclareMathOperator{\SDet}{SDet}
\usepackage{tikz}
  \usetikzlibrary{decorations.markings}
\tikzset{->-/.style={decoration={
  markings,
  mark=at position #1 with {\arrow{>}}},postaction={decorate}}}
\usetheme{default}

\title{Exact Results for $\mathcal{N}=1$ Theories of Class $\mathcal{S}_k$}

\author{Thomas~Bourton}
% - Give the names in the same order as the appear in the paper.
% - Use the \inst{?} command only if the authors have different
%   affiliation.

\institute[DESY] % (optional, but mostly needed)
{
  %
  DESY
  %University of Somewhere
  %\and
  %\inst{2}%
  %Department of Theoretical Philosophy\\
  %University of Elsewhere
  }
% - Use the \inst command only if there are several affiliations.
% - Keep it simple, no one is interested in your street address.

\date{17/10/2019}
% - Either use conference name or its abbreviation.
% - Not really informative to the audience, more for people (including
%   yourself) who are reading the slides online

%\subject{Theoretical Computer Science}
% This is only inserted into the PDF information catalog. Can be left
% out. 

% If you have a file called "university-logo-filename.xxx", where xxx
% is a graphic format that can be processed by latex or pdflatex,
% resp., then you can add a logo as follows:

% \pgfdeclareimage[height=0.5cm]{university-logo}{university-logo-filename}
% \logo{\pgfuseimage{university-logo}}

% Delete this, if you do not want the table of contents to pop up at
% the beginning of each subsection:
\AtBeginSubsection[]
{
  \begin{frame}<beamer>{Outline}
    \tableofcontents[currentsection,currentsubsection]
  \end{frame}
}

% Let's get started
\begin{document}
\begin{frame}
  \titlepage
\end{frame}

\begin{frame}{Outline}
  \tableofcontents
  % You might wish to add the option [pausesections]
\end{frame}

% Section and subsections will appear in the presentation overview
% and table of contents.
\section{Introduction}
\subsection{Motivation: Exact results for $\mathcal{N}=1$ Theories?}
\begin{frame}{Motivation}
\begin{block}{Ambition: Understand the non-perturbative dynamics of QFTs} This is a \textbf{hard} problem (QCD confinement, Yang-Mills mass-gap, etc) $\implies$ consider simplified models $\implies$ \testbf{supersymmetry}
\begin{itemize}
    \item For $\mathcal{N}\geq2$ a lot is understood $\implies \boxed{\text{Exact results!}}$
    \item But, $\mathcal{N}\geq2$ is far away from `real-world physics'
\end{itemize}
  $\mathcal{N}=1$ is a lot closer e.g. $\mathcal{N}=1$ SQCD exhibits:
  \begin{itemize}
      \item Color confinement
      \item Chiral symmetry breaking
    \item Conformal phase
  \end{itemize}
    $\mathcal{N}=1$ SUSY $\implies$ more control over theory than $\mathcal{N}=0$ - still potentially constraining enough to understand the dynamics non-perturbatively.\newline
    \end{block}
\end{frame}
\begin{frame}{Exact Results}
By "exact result" we mean any quantity that we compute that is valid for any value of the parameter space. They give us a window into the non-perturbative dynamics of a theory. Some examples for $\mathcal{N}=2$
\begin{itemize}
      \item{Moduli space $\mathbf{M}$ of SUSY vacua
      \begin{itemize}
          \item Coulomb Branch $\textbf{CB}$ - Seiberg-Witten Theory: Exact computation of low-energy effective action
          \item Higgs branch $\mathbf{HB}$ - Quantum mechanically exact
      \end{itemize}}
          \item {
       Exact evaluation (localisation) of full path integral on various manifolds  e.g. $\mathbb{S}^4$, $\mathbb{S}^1\times\mathbb{S}^3$. \color{blue}[Pestun '07] [Romelsberger '05] [Kinney, Maldacena, Minwalla, Raju '05] [...]
      }
      \item{Nekrasov's instanton counting  \color{blue}[Nekrasov '02]\color{black}: path integral $Z_{\text{inst}}$ over all instanton configurations}
      \item{2d/4d correspondence(s), e.g. AGT, TQFT-Index, etc}
\end{itemize}
\begin{center}
\boxed{\text{What about exact results for $\mathcal{N}=1$?}}
\end{center}
\end{frame}

\subsection{Review: Class $\mathcal{S}$}
\begin{frame}{Class $\mathcal{S}$ Theories}
\begin{block}{A large class of 4d $\mathcal{N}=2$ theories: \color{blue} [Gaiotto '09] [Gaiotto, Moore, Neitzke '09]}
\begin{itemize}
  \itemsep1em
  \item{6d $\mathcal{N}=(2,0)$ SCFT (stack of $N$ M5 branes) on $\mathbb{R}^4\times\mathcal{C}$ where $\mathcal{C}$ is a compact Riemann surface
      \begin{center}
        \begin{tabular}{|c|c|c|c|c|c|c|c|c|c|c|c|}
               \multicolumn{1}{c|}{} &\multicolumn{4}{c|}{$\mathbb{C}^2$}&\multicolumn{2}{c|}{$\mathcal{C}$}&\multicolumn{4}{c|}{$\mathbb{C}^2_{\perp}$}&\multicolumn{1}{c|}{$\mathbb{S}^1$}\\\hline
             $N$ M5 & --&--&--&--&--&--&$\cdot$&$\cdot$&$\cdot$&$\cdot$&$\cdot$ \\\hline
        \end{tabular}
    \end{center}}
  \item{We take $\text{Area}(\mathcal{C})\to0$ $\implies$ 4d $\mathcal{N}=2$ theory on $\mathbb{R}^4$}
\end{itemize}\end{block}
\begin{block}{$\mathcal{C}$ = four punctured sphere $\Leftrightarrow$ $\mathcal{N}=2$ SQCD}
   \begin{figure}
    \centering
  \begin{tikzpicture}[square/.style={regular polygon,regular polygon sides=4},thick,inner sep=0.1em]
    %gaugenodes
    \node (G11) at (0,0)[square,draw,minimum size=1cm]{$N$};
    \node (G21) at (2,0) [circle,draw,minimum size=1cm]{$N$};
    \node (G31) at (4,0) [square,draw,minimum size=1cm]{$N$};
    \draw (G11.0) to (G21.180);
    \draw (G21.0) to (G31.180);
    \draw (-3,0) ellipse (1.5cm and 1cm);
    \fill (-2.6,0.5) circle [radius=1.5pt];  
    \fill (-3.4,0.5) circle [radius=1.5pt];  
    \fill (-4,-0.3) circle [radius=1.5pt]; 
    \draw (-4,-0.3) circle [radius=3pt];
    \fill (-2,-0.3) circle [radius=1.5pt]; 
    \draw (-2,-0.3) circle [radius=3pt];
    \node at (-1,0) {$\Leftrightarrow$};
    \end{tikzpicture}
\end{figure}
\end{block}
\end{frame}

\subsection{Class $\mathcal{S}_k$}
\begin{frame}{Class $\mathcal{S}_k$}
    \begin{block}{Natural generalisation of Class $\mathcal{S}$}
    Instead: Compactify a 6d $\mathcal{N}=(1,0)$ theory on $\mathcal{C}$. 
    \end{block}
    \begin{itemize}
      \itemsep1em
        \item {The 6d $\mathcal{N}=(1,0)$ theory we use is a $\mathbb{Z}_k$ orbifold of the 6d $\mathcal{N}=(2,0)$ theory \color{blue}[Gaiotto, Razamat '15]}
        \item{The $\mathbb{Z}_k$ breaks supersymmetry by $1/2$, leaving $\mathcal{N}=1$ in 4d}
    \end{itemize}
    \begin{center}
        \begin{tabular}{|c|c|c|c|c|c|c|c|c|c|c|c|}
               \multicolumn{1}{c|}{} &\multicolumn{4}{c|}{$\mathbb{C}^2$}&\multicolumn{2}{c|}{$\mathcal{C}$}&\multicolumn{4}{c|}{$\mathbb{C}^2_{\perp}$}&\multicolumn{1}{c|}{$\mathbb{S}^1$}\\\hline
             $N$ M5 & --&--&--&--&--&--&$\cdot$&$\cdot$&$\cdot$&$\cdot$&$\cdot$ \\\hline
             $\mathbb{Z}_k$ &$\cdot$&$\cdot$&$\cdot$&$\cdot$&$\cdot$&$\cdot$&$\times$&$\times$&$\times$&$\times$&$\cdot$ \\\hline
        \end{tabular}
    \end{center}
    \begin{equation*}
        \mathbb{Z}_k: (v_{\perp},w_{\perp})\mapsto (e^{2\pi i/k}v_{\perp},e^{-2\pi i/k}w_{\perp})
    \end{equation*}
        \begin{itemize}
        \item {Class of 'nice' 4d $\mathcal{N}=1$ theories again classified by punctured Riemann surfaces $\mathcal{C}$}
    \end{itemize}
\end{frame}
\begin{frame}{Example}
    \begin{block}{$\mathcal{C}=$ four punctured sphere}
    \end{block}
\begin{itemize}
\itemsep1em
    \item{For $k=2$ - $\mathcal{N}=1$ superconformal quiver
\begin{figure}
\centering
  \begin{tikzpicture}[square/.style={regular polygon,regular polygon sides=4},thick,inner sep=0.1em]
    %gaugenodes
    \node (G11) at (0,0)[square,draw,minimum size=1cm]{$N$};
    \node (G21) at (2,0) [circle,draw,minimum size=1cm]{$N$};
    \node (G31) at (4,0) [square,draw,minimum size=1cm]{$N$};
    \node (G12) at (0,-2)[square,draw,minimum size=1cm]{$N$};
    \node (G22) at (2,-2) [circle,draw,minimum size=1cm]{$N$};
    \node (G32) at (4,-2) [square,draw,minimum size=1cm]{$N$};
    %down chirals phi1
    \draw [->-=.5] (G21.260) to (G22.100);
    \draw [->-=.5] (G22.80) to (G21.280);
    \node at (1.7,-1) {$\Phi$};
    %right phi2
    \node at (1,0.3) {$Q$};
    \draw [->-=.5] (G11.0) to (G21.180);
    \draw [->-=.5] (G21.0) to (G31.180);
    \draw [->-=.5] (G12.0) to (G22.180);
    \draw [->-=.5] (G22.0) to (G32.180);
	%left up Phi3
	\draw [->-=.4] (G22.135) to (G11.-45);
	\node at (0.4,-0.8) {$\widetilde{Q}$};
    \draw [->-=.7] (G32.135) to (G21.-45);
    \draw [->-=.7] (G31.225) to (G22.45);
    \draw [->-=.4] (G21.225) to (G12.45);
    \draw (-3,-1) ellipse (1.5cm and 1cm);
    \fill (-2.6,-0.5) circle [radius=1.5pt];  
    \fill (-3.4,-0.5) circle [radius=1.5pt];  
    \fill (-4,-1.3) circle [radius=1.5pt]; 
    \draw (-4,-1.3) circle [radius=3pt];
    \fill (-2,-1.3) circle [radius=1.5pt]; 
    \draw (-2,-1.3) circle [radius=3pt];
    \node at (-1,-1) {$\Leftrightarrow$};
  \end{tikzpicture}
\end{figure}
}
\item{$W\simeq \left(Q\Phi\widetilde{Q}-\widetilde{Q}\Phi Q\right)-\frac{i\tau}{8\pi^2}\tr W^{\alpha}W_{\alpha}$}
\item{$SU(2)_{R_{\mathcal{N}=2}}\times U(1)_{r_{\mathcal{N}=2}}\xrightarrow{\mathbb{Z}_k} U(1)_{r}\times U(1)_t$}
\item{Global Symmetry: $SU(2,2|1)\times U(1)_t\times U(1)_{\gamma}^{k-1}\times U(1)_{\beta}^{k-1}\times SU(N)^{2k}\times U(1)_{\alpha}^2$}
\end{itemize}
\end{frame}

\section{Exact Results for Class $\mathcal{S}_k$}
\subsection{Moduli Space of SUSY Vacua}
\begin{frame}{Moduli space}
\begin{block}{$\mathcal{N}=1$ gauge theories have a \textit{moduli space} of supersymmetric vacua}
   \begin{equation}
       \mathbf{M}=\left\{\Phi,Q,\widetilde{Q}\middle|V(\Phi,Q,\widetilde{Q})=0\right\}/(\text{gauge transformations})
   \end{equation}
K\"ahler manifold $\mathbf{M}$ param'd by top components of $\frac{1}{2}$-BPS multiplets $\overline{\mathcal{B}}_{r(0,0)}$ of $SU(2,2|1)$.\newline 
   For class $\mathcal{S}_k$ we have \textbf{distinct} Higgs and Coulomb branches \color{blue}[TB, A.~Pini, E.~Pomoni (to appear)]\color{black}. 
   \begin{center}
       \boxed{\text{For generic $\mathcal{N}=1$ theories such distinction is not possible!}}
   \end{center}
\end{block}
The non-anomalous $U(1)_r\times U(1)_t$ global symmetries means these branches are separate and cannot mix.
\end{frame}
\begin{frame}{Higgs Branch}
\begin{center}
\begin{tabular}{c|c|c}
Field&$U(1)_r$&$U(1)_t$\\\hline
     $Q$&$2/3$&$1/2$  \\
     $\widetilde{Q}$&$2/3$&$1/2$\\
     $\Phi$&$2/3$&$-1$
\end{tabular}
\end{center}
   \begin{equation*}
 \boxed{\mathbf{HB}=\mathbf{M}|_{\Phi=0}\,,\quad E=\frac{3}{2}r=2q_t}
   \end{equation*}
   \begin{itemize}
    \item Gauge invariants: $Q_i\widetilde{Q}_j$, $\det Q_i$, $\det\widetilde{Q}_i$
       \item  As with $\mathcal{N}=2$ SUSY $\mathbf{HB}$ is \textit{exact}
       \item On the other hand only K\"ahler (rather than hyperK\"ahler).
   \end{itemize}
\end{frame}

\begin{frame}{Coulomb Branch}
   \begin{equation*}
\boxed{ \mathbf{CB}=\mathbf{M}|_{Q=\widetilde{Q}=0}\,,\quad E=\frac{3}{2}r=-q_t}
   \end{equation*}
   \begin{itemize}
    \item Gauge invariants: $u_n=\tr(\Phi_1\dots\Phi_k)^n$ and $B_i=\det \Phi_i$
       \item  As with $\mathcal{N}=2$ Coulomb branch superpotential has quantum corrections encoded by Seiberg-Witten curve $\Sigma\hookrightarrow T^*\mathcal{C}$ fibred over $\mathbf{CB}$
   \end{itemize}
  Curve can be defined for $\mathcal{N}=1$ theories \color{blue}[Intrilgator, Seiberg '94]\color{black}  
   \begin{equation*}
     \Sigma:\quad  z^{kN}+\sum_{l=1}^Nz^{k(N-l)}\phi_{kl}(t;u_n,B_i)=0
   \end{equation*}
   \color{blue}[Coman, Pomoni, Taki, Yagi '15]\color{black}, \color{blue} [TB, Pomoni (to appear)]\color{black}
   \begin{equation*}
       W_{eff}=\frac{i\tau_{eff}}{8\pi}W_{\alpha}W^{\alpha}\,,\quad \tau_{eff}=\frac{\oint_B\lambda}{\oint_A\lambda}\,,\quad \lambda=\frac{v}{t}dt
   \end{equation*}
\end{frame}
\subsection{Supersymmetric Index and Special Limits}
\begin{frame}{Supersymmetric Index ($\mathbb{S}^3\times \mathbb{S}^1$ partition function)}
Pick $\mathcal{Q}=\widetilde{\mathcal{Q}}_{\dot-}\implies$ $\boxed{2\{\mathcal{Q},\mathcal{Q}^{\dagger}\}=E-2j_2-3r/2=0}$
\begin{equation*}
    \mathcal{I}=\Tr (-1)^Fp^{j_1+j_2+\frac{r}{2}-\frac{2}{3}q_t}q^{-j_1+j_2+\frac{r}{2}-\frac{2}{3}q_t}t^{q_t}e^{-2\beta\{\mathcal{Q},\mathcal{Q}^{\dagger}\}}
\end{equation*}
Witten index graded by fugacities for maximal commutant in $SU(2,2|1)\times U(1)_t\times H$
\begin{itemize}
    \item{ Counts all short multiplets modulo recombination to long ones
    \begin{equation*}
    \mathcal{I}(\mathcal{S})=\chi_{\mathcal{S}}\,,\quad \mathcal{I}(\mathcal{A})=\mathcal{I}(\mathcal{S})+\mathcal{I}(\mathcal{S}')=0
\end{equation*}}
\end{itemize}

Use localisation ($k=N=2$)
\begin{equation*}
    \mathcal{I}=(p;p)^2(q;q)^2\int\frac{dz_1dz_2}{(4\pi i)^2}\prod_{i=1}^{k=2}\frac{\Gamma_e(\sqrt{t}z_i^{\pm1})^2\,\Gamma_e(\frac{pq}{t}z_i^{\pm1}z_{i-1}^{\pm1})}{\Gamma(z_i^{\pm2})}
\end{equation*}
2d/4d relation: Equal to 2d TQFT correlator on $\mathcal{C}$ \color{blue}[Gaiotto, Razamat '15]
\end{frame}

\begin{frame}{Localisation}
\begin{block}{Localisation principle:}
We would like to compute observables:
\begin{equation}
\langle\mathcal{O}[\phi]\rangle=\int\limits_{\mathbf{C}(\mathcal{M})}[\mathcal{D}\phi]\,e^{-S[\phi]}\,\mathcal{O}[\phi]
\end{equation}
Assume $\exists\,\mathfrak{Q}$ such that $\mathfrak{Q}S=\mathfrak{Q}\mathcal{O}=0$ \& $\mathfrak{Q}^2=\text{(bosonic symmetry)}$. We can then deform $S\to S+t\mathfrak{Q}V$
\begin{equation}
    \langle\mathcal{O}[\phi]\rangle_t=\int\limits_{\mathbf{C}(\mathcal{M})}[\mathcal{D}\phi]\,e^{-S[\phi]-t\mathfrak{Q}V[\phi]}\,\mathcal{O}[\phi]\,,\quad t\in\mathbb{R}
\end{equation}
$\boxed{\text{The answer is independent of $t$!}}$ we can take $t\to\infty$ $\imples\,\mathfrak{Q}V=0$
\begin{equation}
   \langle\mathcal{O}[\phi]\rangle_t=\int\limits_{\{a\in\mathbf{C}(\mathcal{M})|\mathfrak{Q}V=0\}} da\frac{e^{-S[a]}\,\mathcal{O}[a]}{\SDet\left[\frac{\delta^2 \mathfrak{Q}V[a]}{\delta a^2}\right]} 
\end{equation}
\end{block}
\end{frame}
\begin{frame}{Limits of the Index}
First defined for \textbf{all} $\mathcal{N}=2$ theories \color{blue}[Gadde, Rastelli, Razamat, Yan '11]\color{black}. The limits can be defined for `\textit{all}' class $\mathcal{S}_k$ theories \color{blue}[TB, Pini, Pomoni (to appear)]\color{black}\newline
\begin{itemize}
    \item{ Hall-Littlewood $p,q\to0$, $t$ fixed $(2q_t=E+j_2\,,j_1=0)$\begin{equation*}
        \mathrm{HL}=\Tr_{HL}(-1)^Ft^{q_t}
    \end{equation*}}
    \item{ Coulomb $t,p,q\to0$, $T=pq/t$, $V=p/q$ fixed $(E+2j_2+\frac{r}{2}+\frac{4q_t}{3}=0)$ \begin{equation*}
        \mathcal{I}^C=\Tr_{C} (-1)^FT^{E+j_2}V^{j_1}
    \end{equation*}}            
\end{itemize}
Can also define Macdonald $p/\sqrt{t}\to0$, $q$ fixed and Schur $q=t$ limits.\newline
$\boxed{\text{Such limits }\textbf{do not }\text{exist for generic $\mathcal{N}=1$ theories!} }$
\end{frame}
\begin{frame}{Limits of the Index}
Hall-Littlewood index: counts $Q,\widetilde{Q}, \widetilde{\mathcal{Q}}_{\dot+}\overline{\Phi}=\overline{\lambda}\implies$ these all have $2q_t=E+j_2,j_1=0$ and are Higgs branch $\mathbf{HB}$ type operators (plus the extra fermions). For genus zero Lagrangian theories it turns out that \color{blue}[Gadde, Rastelli, Razamat, Yan '11]\color{black}
\begin{equation*}
    \mathrm{HL}=\mathrm{Hilb}(t;\mathbf{HB})
\end{equation*}
Coulomb index: counts $\Phi\implies$ these have $E+2j_2+\frac{r}{2}+\frac{4q_t}{3}=0$ and are Coulomb branch $\mathbf{CB}$ type operators and we have
\begin{equation*}
    \mathcal{I}^C=\mathrm{Hilb}(T;\mathbf{CB})
\end{equation*}
The Hilbert series of the variety $\mathbf{M}$ counts independent holomorphic functions
\begin{equation*}
    \mathrm{Hilb}(t;\mathbf{M})=\Tr_{\mathbf{M}}t^E\,,\quad  \begin{aligned}\mathrm{Hilb}(t;\mathbb{C}^2)=&1+2t+3t^2=(1-t)^{-2}\\
    &1\,, \, [z_1,z_2]\,,\, [z_1^2,z_2^2,z_1z_2]\end{aligned}
\end{equation*}

\end{frame}
\subsection{Instanton Counting for Class $\mathcal{S}_k$}
\begin{frame}{Instantons for Class $\mathcal{S}_k$}
\begin{block}{Partition function over Instantons}
\begin{itemize}
  \itemsep1em
    \item{$Z_{\text{inst}}=\sum_nq^KZ_K$ played an important role in $\mathcal{N}=2$ $\mathbb{S}^4$ partition function. $Z_{\mathbb{S}^4}\sim\int daZ_{\text{pert}}|Z_{\text{inst}}|^2$}
    \item{Instantons in susy gauge theories embed in string theory $\implies$ can compute of $Z_{\text{inst}}$ for class $\mathcal{S}_k$}
\end{itemize}
\end{block}
\begin{block}{String theory construction}
\begin{itemize}
\itemsep1em
\item{Embed in Type-IIA \color{blue}[Witten '97]\color{black}
    \begin{center}
        \begin{tabular}{|c|c|c|c|c|c|c|c|c|c|c|}
               \multicolumn{1}{c|}{} &\multicolumn{4}{c|}{$\mathbb{C}^2$}&\multicolumn{2}{c|}{$\mathbb{R}^2$}&\multicolumn{1}{c|}{$\mathbb{S}^1$}&\multicolumn{3}{c|}{$\mathbb{R}^3_{\perp}$}\\\hline
             $N$ D4 & --&--&--&--&$\cdot$&$\cdot$&--&$\cdot$&$\cdot$&$\cdot$ \\\hline
             NS5 &--&--&--&--&--&--&$\cdot$&$\cdot$&$\cdot$&$\cdot$ \\\hline
             $\mathbb{Z}_k$ &$\cdot$&$\cdot$&$\cdot$&$\cdot$&$\times$&$\times$&$\cdot$&$\times$&$\times$&$\cdot$ \\\hline
              $K$ D0 &$\cdot$&$\cdot$&$\cdot$&$\cdot$&$\cdot$&$\cdot$&--&$\cdot$&$\cdot$&$\cdot$ \\\hline
        \end{tabular}
    \end{center}}

    \end{itemize}
    \end{block}
\end{frame}

\begin{frame}{ADHM Construction}
\begin{equation*}
   \boxed{ \text{$K$ instantons in a D$p$-brane $\equiv$ $K$} D$(p-4)$-branes}
\end{equation*}
    \begin{figure}
  \centering
    \begin{tikzpicture}[thick]
\usetikzlibrary{shapes.geometric}
\usetikzlibrary{decorations.markings}
\usetikzlibrary{decorations.pathmorphing}
\tikzset{snake it/.style={decorate, decoration=snake}}
  \draw (1.3,-2) -- (1.3,0.25);
  \draw (-1.3,-2) -- (-1.3,0.25);
  
  \draw (1.3,-0.15) -- (3.2,-0.15);
  \draw (1.3,-0.25) -- (3.2,-0.25);
  \draw (1.3,-0.35) -- (3.2,-0.35);
  \draw (1.3,-0.45) -- (3.2,-0.45);
  
  \draw (-1.3,-0.15) -- (-3.2,-0.15);
  \draw (-1.3,-0.25) -- (-3.2,-0.25);
  \draw (-1.3,-0.35) -- (-3.2,-0.35);
  \draw (-1.3,-0.45) -- (-3.2,-0.45);
  
  \filldraw [color=red](-0.9,-0.32) circle (0.8pt);
  \filldraw [color=red](-0.5,-0.43) circle (0.8pt);
  \filldraw [color=red](-0.2,-0.25) circle (0.8pt);
  \filldraw [color=red](0.86,-0.39) circle (0.8pt);
  \draw (-1.3,-0.2) -- (1.3,-0.2);
  \draw (-1.3,-0.3) -- (1.3,-0.3);
  \draw (-1.3,-0.4) -- (1.3,-0.4);
  \draw (-1.3,-0.5) -- (1.3,-0.5);
  
  \draw (-1.3,-1.65) -- (-3.2,-1.65);
  \draw (-1.3,-1.55) -- (-3.2,-1.55);
  \draw (-1.3,-1.45) -- (-3.2,-1.45);
  \draw (-1.3,-1.35) -- (-3.2,-1.35);
  
  \draw (1.3,-1.65) -- (3.2,-1.65);
  \draw (1.3,-1.55) -- (3.2,-1.55);
  \draw (1.3,-1.45) -- (3.2,-1.45);
  \draw (1.3,-1.35) -- (3.2,-1.35);
    
  \draw (-1.3,-1.3) -- (1.3,-1.3);
  \draw (-1.3,-1.5) -- (1.3,-1.5);
  \draw (-1.3,-1.6) -- (1.3,-1.6);
  \draw (-1.3,-1.4) -- (1.3,-1.4);
  \filldraw [color=red](-0.9,-1.48) circle (0.8pt);
  \filldraw[color=red] (-0.5,-1.37) circle (0.8pt);
  \filldraw [color=red](-0.2,-1.55) circle (0.8pt);
  \filldraw[color=red] (0.86,-1.41) circle (0.8pt);
  
  \node at (1.3,-2.2) {NS5};
  \node at (-1.3,-2.2) {NS5};
  \node at (2.2,0.15) {$N_R$ D4};
  \node at (-2.2,0.15) {$N_L$ D4};
  \node at (0,0.1) {$N$ D4 + \color{red}$K$ D0};
  \draw [dashed] (3.2,-0.9) to (-3.2,-0.9) node[left] {$\mathbb{Z}_k$};
  
  \end{tikzpicture}
\end{figure}
\begin{itemize}
\item{D$0$ Higgs branch = ADHM moduli space of instantons
\begin{equation*}
    \mathbf{M}^{\text{D}4}_{\text{inst}}\cong\mathbf{HB}^{\text{D}0}
\end{equation*}\text{\color{blue}[Douglas '95] [Witten '95] [Douglas '96]}}
    \item {Reduction to 0d of 2d theory with at least one supercharge $\mathcal{Q}_+$}
\end{itemize}
\end{frame}

\begin{frame}{Computation of $Z_{\text{inst}}$}
\begin{itemize}
    \item {Partition function $Z_{\text{D}0}$ for sigma model living on $K$ D$0$-branes equals $Z_K$}
   % \item {\alert{Fact}: Higgs branch is invariant under circle compactification (When we have susy)}
    %\item{Lift the 0d theory to a 2d theory on $\mathbb{S}^1\times\mathbb{S}^1$ (2$\times $T-duality: D$0$ $\to$ D$2$). Radii $\beta_1$, $\beta_2$}
    \item{$\mathbb{S}^1\times\mathbb{S}^1$ partition function is supersymmetric index of the 2d theory!
        \begin{equation*}
        Z_K=Z_{\text{D}0}=\lim_{\beta_1,\beta_2\to0}\mathcal{I}_{\text{D}2}
    \end{equation*}}
    \item{$\mathcal{Q}_+$ is \alert{preserved} we are counting gauge invariant operators on $\mathbb{R}^2/\mathbb{Z}_k$ which have $\{\mathcal{Q}_+,\mathcal{Q_+}^{\dagger}\}=0$}
    \item{We can play the same games and compute 
        \begin{equation*}
        \mathcal{I}_{\text{D$2$}}=\Tr(-1)^Fq^{H_-}v^{2j_1}t^{2j_D}\mathbf{x}^{\mathbf{f}}
    \end{equation*} by projecting onto the $\mathbb{Z}_k$-invariant states \color{blue}[TB, Pomoni '17]\color{black}}
\end{itemize}
\end{frame}

\begin{frame}{Orbifolding to Class $\mathcal{S}_k$}
        \begin{center}
        \boxed{
       \mathcal{I}_{\text{D2}}=\int \left[d\mu_G(\mathbf{z})\right]\mathrm{PE}\left[\sum_{\{\text{multiplets}\}}i^{\text{orb}}_{multiplet}(q,\mathbf{z},\dots)\right] }
        \end{center}
\begin{itemize}
\item{Take the limit}
    \end{itemize}
\begin{equation*}
\begin{aligned}
    \lim_{\beta_1,\beta_2\to0} \mathcal{I}_{\text{D2}}\propto\prod_{i=1}^k\int \prod_{I=1}^{K_i}du_{i,I}\prod_{I=1}^{K_i}\frac{u_{ii,IJ}'\prod_{j\neq i}\prod_{J=1}^{K_j}\left(u_{ij,IJ}-2\epsilon_+\right)}{\prod_{j=1}^k\prod_{J=1}^{K_j}\left(u_{ij,IJ}+\epsilon_1\right)\left(u_{ij,IJ}+\epsilon_2\right)}&\\
    \times\prod_{j=1}^k\prod_{I=1}^{K_i}\prod_{A=1}^N\frac{\left(u_{i,I}-\widetilde{m}_{L,j,A}\right)\left(u_{i,I}-\widetilde{m}_{R,j,A}\right)}{\left(u_{i,I}-\widetilde{a}_{j,A}-\epsilon_+\right)\left(u_{i,I}-\widetilde{a}_{j,A}+\epsilon_+\right)}&
    \end{aligned}
\end{equation*}
\begin{itemize}
\item{We expect this to be equal to the integration over the moduli space of $\{K_1,K_2,\dots,K_k\}$ instantons for this $\mathcal{N}=1$ theory}
    \end{itemize}
\end{frame}


% Placing a * after \section means it will not show in the
% outline or table of contents.
\section{Conclusions and Future Directions}

\begin{frame}{Conclusions and Future Directions}
  \begin{itemize}
\item{$Z_{\text{inst}}$ equal to $\mathcal{W}_{kN}$ conformal blocks - AGT?}
  \end{itemize}
\end{frame}


\end{document}


