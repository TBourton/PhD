\documentclass[main.tex]{subfiles}
\begin{document} 
\section*{Abstract}
\section{Introduction}
4d theories with $\mathcal{N}\geq1$ supersymmeries on manifolds of the form $S^1\times L(q,p)$ where $L(r,p)=S^3/\mathbb{Z}_p$ is a Lens space have been stuided in \cite{Benini:2011nc,Alday:2013rs,Razamat:2013jxa,Fluder:2017oxm,Nishioka:2014zpa,Razamat:2013opa}. It is also possible to study 3d theories with $\mathcal{N}\geq2$ on said Lens spaces. 
\section{The Lens spaces $L(q_1,q_2,q_3;p)$}
Let $p,q_1,q_2,q_3$ be integers such that $gcd(q_i,p)=1$ for all $i=1,2,3$. The Lens space $L(q_1,q_2,q_3;p)\subset\mathbb{C}^3$ is the quotient $S^5/\mathbb{Z}_p$ generated by the free $U(1)^3\supset\mathbb{Z}_p\acts\mathbb{C}^3$ action
\begin{equation}\label{eqn:Lensaction}
\mathbb{Z}_p:(z_1,z_2,z_3)\mapsto\left(\gamma^{q_1}z_1,\gamma^{q_2}z_2,\gamma^{q_3}z_3\right)
\end{equation}
where $\gamma=e^{2\pi\iu/p}$ and $|z_1|^2+|z_2|^2+|z_3|^2=1$. When there is no ambiguity, in order to save on notation, we will often write $L_p\equiv L(q_1,q_2,q_3;p)$. The Lens space satisfies:
\begin{equation}
\pi_i\left(L_p\right)=\begin{cases}
\mathbb{Z}_p & i=1\\
\pi_{i}\left(S^5\right)&i\geq2\,.
\end{cases}\,,
\quad H_i\left(L_p\right)=\begin{cases}
\mathbb{Z} & i=0,5\\
\mathbb{Z}_p&i=1,3\\
0&i\neq0,1,3,5
\end{cases}\,.
\end{equation}
Notice that the above depend only on $p$ and are independent of the $q_i$. They are of physical interest because $\pi_1(L_p)=\mathbb{Z}_p$ is non-trivial and therefore the Lens space index is sensitive to the line operator spectrum of the theory.

\section{The $S^1\times L(q_1,q_2,q_3;p)$ partition function of 6d $\mathcal{N}=(2,0)$ SCFTs}
\label{Sec:Lensspace}

\subsection{Lens space index}
In this paper we will be interested in theories with $\mathcal{N}=(2,0)$ supersymmetry in six dimensions. These theories (at the level of the local operator spectrum) are in one to one correspondence with the finite subgroups of $SU(2)$ \cite{Witten:1995zh,Hanany:2000fq}. It is therefore common to label them of type $\mathfrak{g}=ADE$. Compactification of the theory of type $\mathfrak{g}$ on a circle with radius $\beta\to0$ gives rise to the 5d $\mathcal{N}=2$ SYM theory with gauge algebra $\mathfrak{g}$.  It is conjectured that (atleast at the level of BPS states) that the 6d $\mathcal{N}=(2,0)$ theory of type $\mathfrak{g}$ on $S_{\beta}^1\times S^5$ is equivalent to the 5d $\mathcal{N}=2$ theory with gauge algebra $\mathfrak{g}$ and gauge coupling $g^2_{\text{YM}}=2\pi\beta$. It is therefore believed that the $S^5$ partition function for the 5d theory is equal to the superconformal index of the $\mathcal{N}=(2,0)$ theory. The parameters $\omega_i$, $\mu/2$ are identified with the squashing parameters of the $S^5$ and the hypermultiplet mass \cite{Kim:2012ava} respectively.

The $\mathcal{N}=(2,0)$ superconformal algebra is $\mathfrak{osp}(8|4)$. Representations of $\mathfrak{osp}(8|4)$ are labelled by the Cartans of the maximal bosonic subalgebra $\mathfrak{so}(2,6)\oplus\mathfrak{usp}(4)$. The Cartans are $(E,h_1,h_2,h_3,R_1,R_2)$, where $E$ corresponds to dilatations, $(h_1,h_2,h_3)$ to 2-plane rotations in $\mathbb{R}^6$ and $R_1,R_2$ to $\mathfrak{usp}(4)\iso\mathfrak{so}(5)$ Cartans. 
The theory has $16$ Poincar\'e supercharges $\mathcal{Q}^{R_1R_2}_{h_1h_2h_3}$ with $-8h_1h_2h_3=2E=1$ and $16$ conformal supercharges with $-8h_1h_2h_3=2E=-1$.

We will define the superconformal index with respect to the supercharge $\mathcal{Q}:=\mathcal{Q}^{++}_{---}$ and its conjugate $\mathcal{Q}^{\dagger}=\mathcal{Q}^{--}_{+++}$ which have $\frac{R_1+R_2}{2}=E=-h_i=\frac{1}{2}$ and $\frac{R_1+R_2}{2}=E=-h_i=-\frac{1}{2}$ respectively.
Under the action \eqref{eqn:Lensaction} the supercharges transform as
\begin{equation}
\mathcal{Q}^{R_1R_2}_{h_1h_2h_3}\mapsto\gamma^{q_1h_1+q_2h_2+q_3h_3}\mathcal{Q}^{R_1R_2}_{h_1h_2h_3}
\end{equation}
therefore, in order to preserve $\mathcal{Q},\mathcal{Q}^{\dagger}$, we have to specialise to
\begin{equation}\label{eqn:qspecialisation}
q_1+q_2+q_3=0\bmod 2p\,,
\end{equation}
for simplicity in this paper we will drop the modulo condition and simply write $q_1+q_2+q_3=0$, this also means that we have to take $gcd(q_1+q_2,p)=1$. In that case the action \eqref{eqn:Lensaction} preserves $\mathcal{Q}_{+++}^{R_1R_2}$ and their conjugates. Note that one could, in order to preserve more supersymmetries, also decide to turn on non-trivial background R-symmetry currents, however we will not pursue this here.

The Lens space index is then defined to be \cite{Bhattacharya:2008zy,Kim:2012ava,Kinney:2005ej,Lockhart:2012vp,Romelsberger:2005eg,Bak:2016vpi}
\begin{equation}
\begin{aligned}\label{eqn:Lensindex}
\mathcal{I}_{p}\left(\qbf_1,\qbf_2,\qbf_3,\pbf\right)=&\Tr_{L_p}(-1)^Fe^{-\beta\left(\delta+h_1+h_2+h_3+\frac{3R_1+3R_2}{2}\right)-\beta(a_1h_1+a_2h_2+a_3h_3)-\beta\mu\frac{R_2-R_1}{2}}\\
=&\Tr_{L_p}(-1)^Fe^{-\beta\delta}\qbf_1^{h_1+\frac{R_1+R_2}{2}}\qbf_2^{h_2+\frac{R_1+R_2}{2}}\qbf_3^{h_3+\frac{R_1+R_2}{2}}\pbf^{R_2-R_1}
\end{aligned}
\end{equation}
with $a_1+a_2+a_3=0$, or equivalently $\qbf_1\qbf_2\qbf_3=e^{-3\beta}$. The trace is taken over the Hilbert space on $L_p=L(q_1,q_2,q_3;p)$ in the radial quantisation. We also defined 
\begin{align}\label{eqn:indexparamas}
\qbf_i:=e^{-\beta(a_i+1)}=e^{-\beta\omega_i}\,,\quad\pbf:=e^{-\beta\mu/2}\,.
\end{align}
Here the $a_i=\omega_i-1$ are related to $U(1)^3\acts\mathbb{C}^3$ corresponding to
\begin{equation}\label{eqn:u13action}
(z_1,z_2,z_3)\mapsto(e^{\iu a_1}z_1,e^{\iu a_2}z_2,e^{\iu a_3}z_3)\,.
\end{equation}
The Lens space index \eqref{eqn:Lensindex} receives contribution only from states satisfying
\begin{equation}
\delta:=2\left\{\mathcal{Q},\mathcal{Q}^{\dagger}\right\}=E-h_1-h_2-h_3-2R_1-2R_2=0\,.
\end{equation}
Since \eqref{eqn:Lensindex} receives contribution only from states with $\delta=0$ it is; considered as a formal power series in the $\qbf_i$ \& $\pbf$, independent of $\beta$.  

\subsection{Casimir energy}
Recall that the $p=1$ superconforma index \eqref{eqn:Lensindex} is expected to be equal to the $S^1\times L_1$ partition function $\mathcal{Z}_{p=1}$ up to an overall Casimir energy factor \cite{Benini:2011nc,Bobev:2015kza,Kim:2012qf,Kim:2012ava}. We expect that such a structure also holds for $p>1$, namely,
\begin{equation}
\mathcal{Z}_p(\qbf_1,\qbf_2,\qbf_3,\pbf)=e^{-\beta E_p(\mathfrak{g})}\mathcal{I}_p(\qbf_1,\qbf_2,\qbf_3,\pbf)\,.
\end{equation}
For $p=1$ it is conjectured that the Casimir energy $E_{p=1}(\mathfrak{g})$ is equal to an equivariant integral of the anomaly polynomial of the $\mathcal{N}=(2,0)$ theory of type $\mathfrak{g}$ \cite{Bobev:2015kza}
\begin{equation}\label{eqn:Casimir}
E_{p=1}(\mathfrak{g})=-\int A_8(\mathfrak{g})\,.
\end{equation}
For $\mathfrak{g}=\mathfrak{u}(1)$ (we use this notation to denote the free theory on a single M5-brane) and $\mathfrak{g}=ADE$ the anomaly polynomial is respetively given by \cite{Yi:2001bz,Intriligator:2000eq,Harvey:1998bx}
\begin{align}
&A_8(\mathfrak{u}(1))=\frac{1}{48}\left[p_2(NM)-p_2(TM)+\frac{1}{4}(p_1(NM)-p_1(TM))^2\right]\,,\\
&A_8(\mathfrak{g})=r_{\mathfrak{g}}A_8(\mathfrak{u}(1))+d_{\mathfrak{g}}h_{\mathfrak{g}}^{\vee}\frac{p_2(NM)}{24}\,.
\end{align}
Here $r_{\mathfrak{g}}$, $d_{\mathfrak{g}}$, $h_{\mathfrak{g}}^{\vee}$ denote the rank, dimension and dual coexter number of $\mathfrak{g}$; for $\mathfrak{g}=A_{N-1}$ these are $N-1$, $N^2-1$ and $N$ respectively. $NM$ and $TM$ denote the normal and tangent bundles to the six-manifold $M=S^1\times S^5$ and $p_j(V)$ denotes the $j^{\text{th}}$ Pontryagin class of the bundle $V$. The integral \eqref{eqn:Casimir} may be promoted to an equivariant integral with respect to the $U(1)^4\acts S^1\times L_{p=1}$ action \cite{Bobev:2015kza}
\begin{align}
&E_{p=1}(\mathfrak{u}(1))=-\frac{1}{16\beta}-\frac{1}{48\omega_1\omega_2\omega_3}\left[
   \frac{1}{8}\prod_{l_1,l_2\in\{1,-1\}}\left(\omega_1+l_1\omega_2+l_2\omega_3\right) +2\mu^4- 
  \mu^2 \sum_{i=1}^3\omega_i^2\right]\,,\\
&E_{p=1}(\mathfrak{g})=r_{\mathfrak{g}}E_{p=1}(\mathfrak{u}(1))-d_{\mathfrak{g}}h_{\mathfrak{g}}^{\vee}\frac{\left(\frac{9}{4}-\beta^2\mu^2\right)^2}{24\beta}\,,
\end{align}
we also still have the condition $\sum_{i=1}^3\omega_i=3$.
Assuming that $|a_i|<1$ we can therefore write
\begin{equation}
e^{-\beta E_{p=1}(\mathfrak{g})}=e^{-\beta\frac{-2r_{\mathfrak{g}}+d_{\mathfrak{g}}h^{\vee}_{\mathfrak{g}}\left(\frac{9}{4}-\mu^2\right)^2}{24}}e^{-\beta r_{\mathfrak{g}}}
\end{equation}
\subsection{The Lens space index of the free tensor multiplet}
Let us begin with the most simple example - the free tensor multiplet. In this case we can actually letter counting so it should serve as a good example.
The Lens space index is given by projecting onto $\mathbb{Z}_p$ invariant quantities. We can write
\begin{equation}\label{eqn:freetensor}
\mathcal{I}_{p}(\qbf_1,\qbf_2,\qbf_3,\pbf)=\PE\left[\frac{1}{|\mathbb{Z}_p|}\sum_{\gamma\in\mathbb{Z}_p}i_p(\qbf_1,\qbf_2,\qbf_3,\pbf;\gamma)\right]\,.
\end{equation}
$i_p$ is the single letter index and is computed by enumerating the single letters, weighted by the orbifold action \eqref{eqn:Lensaction}. These are listed in Table \ref{fig:freetensor}. The sum over $\mathbb{Z}_p$ projects onto $\mathbb{Z}_p$-invariant quantities. After enforcing the condition \eqref{eqn:qspecialisation} the Lens space index is given by
\begin{equation}\label{eqn:freetensorsing}
i_p(\qbf_1,\qbf_2,\qbf_3,\pbf;\gamma)=\frac{\left(\pbf+\pbf^{-1}\right)\sqrt{\qbf_1\qbf_2\qbf_3}-\gamma^{q_1+q_2}\qbf_1\qbf_2-\gamma^{-q_2}\qbf_1\qbf_3-\gamma^{-q_1}\qbf_2\qbf_3+\qbf_1\qbf_2\qbf_3}{\left(1-\gamma^{q_1}\qbf_1\right)\left(1-\gamma^{q_2}\qbf_2\right)\left(1-\gamma^{-q_1-q_2}\qbf_3\right)}\,.
\end{equation}
\begin{table}
\centering
\begin{tabular}{|c||c|c|c|c|c|c|c|} 
\hline
\textrm{Letters}  & $E$ & $h_1$ & $h_2$ & $h_3$ & $R_1$ & $R_2$ &$i_p(\qbf_1,\qbf_2,\qbf_3,\pbf;\gamma)$\\ 
  \hline\hline
 $\phi$ & $2$ & $0$ & $0$ & $0$ & $1$ & $0$ & $\pbf^{-1}\sqrt{\qbf_1\qbf_2\qbf_3}$\\ 
 \hline
 $\phi$ & $2$ & $0$ & $0$ & $0$ & $0$ & $1$ & $\pbf\sqrt{\qbf_1\qbf_2\qbf_3}$\\ 
 \hline
   $\lambda_{++-}^{++}$ & $5/2$ & $1/2$ & $1/2$ & $-1/2$ & $1/2$ & $1/2$& $-\gamma^{\frac{q_1+q_2-q_3}{2}}\qbf_1\qbf_2$\\ 
 \hline
   $\lambda_{+-+}^{++}$ & $5/2$ & $1/2$ & $-1/2$ & $1/2$ & $1/2$ & $1/2$& $-\gamma^{\frac{q_1-q_2+q_3}{2}}\qbf_1\qbf_3$\\ 
 \hline
    $\lambda_{-++}^{++}$ & $5/2$ & $-1/2$ & $1/2$ & $1/2$ & $1/2$ & $1/2$&$-\gamma^{\frac{-q_1+q_2+q_3}{2}}\qbf_2\qbf_3$\\ 
 \hline\hline
      $\partial\lambda=0$& $7/2$ & $1/2$ & $1/2$ & $1/2$ & $1/2$ & $1/2$ & $\gamma^{\frac{q_1+q_2+q_3}{2}}\qbf_1\qbf_2\qbf_3$\\ 
 \hline\hline
   $\partial_{z_{i=1,2,3}}$ & $1$ & $1,0,0$ & $0,1,0$ & $0,0,1$ & $0$ & $0$ &$\gamma^{q_1}\qbf_1$, $\gamma^{q_2}\qbf_2$, $\gamma^{q_3}\qbf_3$\\ 
 \hline
\end{tabular}
\caption{\textit{The abelian tensor multiplet has a scalar in the fundamental of $\mathfrak{so}(5)$, 16 fermions $\lambda^{R_1R_2}_{h_1h_2h_3}$ with $8h_1h_2h_3=-1$ and a self-dual 3-form $H=\star H$.}}
\label{fig:freetensor}
\end{table}
In order to evaluate the sum \eqref{eqn:freetensor} we have to evaluate sums of the form
\begin{equation}\label{eqn:testsum}
S(b)=\frac{1}{p}\sum_{\gamma\in\mathbb{Z}_p}\sum_{n_1,n_2,n_3=0}^{\infty}\gamma^{q_1(n_1-n_3)+q_2(n_2-n_3)+b}\qbf_1^{n_1}\qbf_2^{n_2}\qbf_3^{n_3}\,,
\end{equation} 
with, $b\in\mathbb{Z}$. The sum can be reduced to
\begin{equation}\label{eqn:testsumres}
S(b)=\frac{\sum_{(n_1,n_2,n_3)\in s_{b}(\eta)}\qbf_1^{n_1}\qbf_2^{n_2}\qbf_3^{n_3}}{(1-\qbf_1^p)(1-\qbf_2^p)(1-\qbf_3^p)}\,,
\end{equation}
where
\begin{equation}\label{eqn:orbset}
s_{b}(\eta)=\left\{(n_1,n_2,n_3)|\eta+b=0\bmod p\,,\, 0\leq n_1,n_2,n_3\leq p-1\right\}\subset\mathbb{N}^3\,,
\end{equation}
and $\eta:=q_1n_1+q_2n_2+q_3n_3$. Unfortunately, we were unable to further reduce the set $s_b$ for general values of the $q_i$, $p$.
The proof goes as follows: For $p=1$ the sum \eqref{eqn:testsum} is simply the generating function for triplets of natural numbers $(n_1,n_2,n_3)\in\mathbb{N}^3$. On the other hand for $p>1$ \eqref{eqn:testsum} is the generating function for triplets $(n_1,n_2,n_3)\in\mathbb{N}^3$ with the orbifold constraint $\eta+b=0\bmod p$. This space is simply given by 
\begin{equation}
S=\vecspan_{p\mathbb{N}}\{(n_1,n_2,n_3)|(n_1,n_2,n_3)\in s_{b}(\eta)\}\,.
\end{equation}
The generating function for $S$ is given precisely by \eqref{eqn:testsumres}. One can also easily verify \eqref{eqn:testsumres} via expansion of \eqref{eqn:testsum} in \texttt{Mathematica} for given any given values of the $q_i$, $p$ and $b$.  
\begin{comment}
\begin{equation}
n_1-n_3=[[b_1]]_{p/q_1}+l_1p/q_1\,,\quad n_2-n_3=[[b_2]]_{p/q_2}+l_2p/q_2\,,
\end{equation}
where,
\begin{equation}\label{eqn:orbcond}
[[x]]_{\ell}:=\left\{
\#\in\mathbb{Z} \,\middle|\text{ $0\leq \#\leq \ell-1$ and $\#=x\bmod \ell$}\right\}\,.
\end{equation} 
We split the sum into four pieces $S(b_1,b_2)=S_{++}+S_{+-}+S_{-+}+S_{--}$ depending on $\sign l_1, \sign l_2$. When $l_1,l_2\geq0$ we can solve for $n_1$ and $n_2$ and then 
\begin{equation}
S_{++}=\sum_{l_1,l_2,n_3\geq0}\qbf^{l_1p/q_1+n_3+[[b_1]]_{p/q_1}}_1\qbf_2^{l_2p/q_2+n_3+[[b_2]]_{p/q_2}}\qbf_3^{n_3}\,.
\end{equation} 
When $l_1<0$ and $l_2\geq0$ we can solve $n_3=n_1-l_1p/q_1-[[b_1]]_{p/q_1}\geq0$ and therefore $n_2=l_2p/q_2+n_1-l_1p/q_1+[[b_2]]_{p/q_2}-[[b_1]]_{p/q_1}\geq0$ so
\begin{equation}
S_{-+}=\sum_{n_1,\tilde{l}_1,l_2\geq0}\qbf_1^{n_1}\qbf_2^{l_2p/q_2+n_1+(\tilde{l}_1+1)p/q_1+[[b_2]]_{p/q_2}-[[b_1]]_{p/q_1}}\qbf_3^{n_1+(\tilde{l}_1+1)p/q_1-[[b_1]]_{p/q_1}}\,.
\end{equation}
where $\tilde{l}_1=-l_1-1\geq0$.  Similarly for $l_1\geq0$, $l_2<0$ we solve $n_3=n_2-l_2p/q_2-[[b_2]]_{p/q_2}\geq0$ and $n_1=l_1p/q_1+n_2-l_2p/q_2+[[b_1]]_{p/q_1}-[[b_2]]_{p/q_2}\geq0$ so
\begin{equation}
S_{+-}=\sum_{n_2,l_1,\tilde{l}_2\geq0}\qbf_1^{l_1p/q_1+n_2+(\tilde{l}_2+1)p/q_2+[[b_1]]_{p/q_1}-[[b_2]]_{p/q_1}}\qbf_2^{n_2}\qbf_3^{n_2+(\tilde{l}_2+1)p/q_2-[[b_2]]_{p/q_2}}\,.
\end{equation}
Now, when both $l_1,l_2<0$, we have to solve $n_3=n_1-l_1p/q_1-[[b_1]]_{p/q_1}=n_2-l_2p/q_2-[[b_2]]_{p/q_2}\geq0$ and so $n_1-n_2=l_1p/q_1-l_2p/q_2+[[b_1]]_{p/q_1}-[[b_2]]_{p/q_2}$. There are therefore two subcases depending on the sign of $l_1p/q_1-l_2p/q_2+[[b_1]]_{p/q_1}-[[b_2]]_{p/q_2}:=l\in\mathbb{Z}$. Case one is $l\geq0$ then we solve $n_1=n_2+l\geq0$, $l_2p/q_2+[[b_2]]_{p/q_2}=-l+l_1p/q_1+[[b_1]]_{p/q_1}$ and we sum over $l,n_2,l_1$. Case two is $l<0$ then we solve $n_2=n_1-l\geq0$, $l_1p/q_1+[[b_1]]_{p/q_1}=l+l_2p/q_2+[[b_2]]_{p/q_2}$ and we sum over $l,n_1,l_2$. Finally
\begin{equation}
S_{--}=\sum_{l,n_2,\tilde{l}_1\geq0}\qbf_1^{n_2+l}\qbf_2^{n_2}\qbf_3^{n_2+l+(\tilde{l}_1+1)p/q_1-[[b_1]]_{p/q_1}}+\sum_{\tilde{l},n_1,\tilde{l}_2\geq0}\qbf_1^{n_1}\qbf_2^{n_1+(\tilde{l}+1)}\qbf_3^{n_1+(\tilde{l}+1)+(\tilde{l}_2+1)p/q_2-[[b_2]]_{p/q_2}}
\end{equation}
Collecting everything we arrive at
\begin{equation}
\begin{aligned}
&S(b_1,b_2)=\frac{1}{(1-\qbf_1\qbf_2\qbf_3)}\left\{\frac{\qbf_1^{[[b_1]]_{p/q_1}}\qbf_2^{[[b_2]]_{p/q_2}}}{\left(1-\qbf_1^{p/q_1}\right)\left(1-\qbf_2^{p/q_2}\right)}+\frac{\qbf_2^{p/q_1+[[b_2]]_{p/q_2}-[[b_1]]_{p/q_1}}\qbf_3^{p/q_1-[[b_1]]_{p/q_1}}}{\left(1-\qbf_2^{p/q_2}\right)\left(1-\qbf_2^{p/q_1}\qbf_3^{p/q_1}\right)}\right.\\
&\left.+\frac{\qbf_1^{p/q_2+[[b_1]]_{p/q_1}-[[b_2]]_{p/q_2}}\qbf_3^{p/q_2-[[b_2]]_{p/q_2}}}{\left(1-\qbf_1^{p/q_1}\right)\left(1-\qbf_1^{p/q_2}\qbf_3^{p/q_2}\right)}+\frac{\qbf_3^{p/q_1-[[b_1]]_{p/q_1}}}{\left(1-\qbf_1\qbf_3\right)\left(1-\qbf_3^{p/q_1}\right)}+\frac{\qbf_2\qbf_3^{p/q_2-[[b_2]]_{p/q_2}+1}}{\left(1-\qbf_2\qbf_3\right)\left(1-\qbf_3^{p/q_2}\right)}\right\}\,.
\end{aligned}
\end{equation}
The function has the correct symmetry properties, namely that it is symmetric under the exchange $\qbf_1\leftrightarrow\qbf_2$ only when $q_1=q_2$ (and $b_1=b_2$) and it is never symmetric under $\qbf_3\leftrightarrow\qbf_1$ or $\qbf_3\leftrightarrow\qbf_2$ for $p>1$. Then 
Of course when $p=1$ we obtain the expression \eqref{eqn:freetensorsing} evaluated at $\gamma=1$.
\end{comment}
Then
\begin{equation}
\begin{aligned}
\frac{1}{p}\sum_{\gamma\in\mathbb{Z}_p}i_p(\qbf_1,\qbf_2,\qbf_3,\pbf;\gamma)=&(\pbf+\pbf^{-1})\sqrt{\qbf_1\qbf_2\qbf_3}S(0)-\qbf_1\qbf_2S(q_1+q_2)\\&-\qbf_1\qbf_3S(q_2)-\qbf_2\qbf_3S(q_1)+\qbf_1\qbf_2\qbf_3S(0)\,.
\end{aligned}
\end{equation}
For example, for $p=3$ and $2q_1=2q_2=-q_3=2$ we have
\begin{equation}
\begin{aligned}
&\frac{1}{3}\sum_{\gamma\in\mathbb{Z}_3}i_3(\qbf_1,\qbf_2,\qbf_3,\pbf;\gamma)=\frac{1}{(1-\qbf_1^3)(1-\qbf_2^3)(1-\qbf_3^3)}\Bigg\{\left(\pbf+\pbf^{-1}+\sqrt{\qbf_1\qbf_2\qbf_3}\right)\sqrt{\qbf_1\qbf_2\qbf_3}\\
&\times\left(1 + \qbf_2 \qbf_3 (\qbf_2 + \qbf_3) + \qbf_1 (\qbf_2^2 + \qbf_2 \qbf_3 + \qbf_3^2) + \qbf_1^2 (\qbf_2 + \qbf_3 + \qbf_2^2 \qbf_3^2)\right)\\
&-(\qbf_2 + \qbf_1 (1 + \qbf_1 \qbf_2^2) + \qbf_3 + 
    \qbf_1 \qbf_2 (\qbf_1 + \qbf_2) \qbf_3 + (\qbf_1^2 + \qbf_1 \qbf_2 + \qbf_2^2) \qbf_3^2) (\qbf_2 \qbf_3 + 
   \qbf_1 (\qbf_2 + \qbf_3))\Bigg\}
\end{aligned}
\end{equation}
\paragraph{$\beta\to0$ limit}
Let us define $Q=e^{-\beta}$. We would like to take the $\beta\to0$ ($Q\to1$) limit. Following \cite{Gadde:2011ia,Dolan:2011rp} we can write the \eqref{eqn:freetensor} as
\begin{equation}
\begin{aligned}
&\mathcal{I}_{p}\left(Q^{\omega_1},Q^{\omega_2},Q^{\omega_3},Q^{\mu}\right)=\prod_{r_1,r_2,r_3=0}^{\infty}\left\{\frac{\prod\limits_{(n_1,n_2,n_3)\in s_{q_1+q_2}(\eta)}\left[\omega_1+\omega_2+\sum_{i=1}^3\omega_i(n_i+pr_i)\right]_Q}{\prod\limits_{(n_1,n_2,n_3)\in s_0(\eta)}\left[\mu+\sum_{i=1}^3\omega_i(n_i+pr_i+\frac{1}{2})\right]_Q}\times\right.\\
&\left.\frac{\prod\limits_{(n_1,n_2,n_3)\in s_{q_2}(\eta)}\left[\omega_1+\omega_3+\sum_{i=1}^3\omega_i(n_i+pr_i)\right]_Q\prod\limits_{(n_1,n_2,n_3)\in s_{q_1}(\eta)}\left[\omega_2+\omega_3+\sum_{i=1}^3\omega_i(n_i+pr_i)\right]_Q}{\prod\limits_{(n_1,n_2,n_3)\in s_0(\eta)}\left[-\mu+\sum_{i=1}^3\omega_i(n_i+pr_i+\frac{1}{2})\right]_Q\prod\limits_{(n_1,n_2,n_3)\in s_0(\eta)}\left[\sum_{i=1}^3\omega_i(n_i+pr_i+1)\right]_Q}\right\}\,,
\end{aligned}
\end{equation}
where $\left[n\right]_Q=(1-Q^n)/(1-Q)$ is the $Q$-number. The $Q$ number satisfies the property that $\lim_{Q}[n]_Q=n$.
Therefore
\begin{equation}
\begin{aligned}
&\lim_{\beta\to0}\mathcal{I}_{p}\left(Q^{\omega_1},Q^{\omega_2},Q^{\omega_3},Q^{\mu}\right)=\frac{\prod\limits_{(n_1,n_2,n_3)\in s_0(\eta)}\Gamma_3\left(\mu+\sum_{i=1}^3\omega_i(n_i+\frac{1}{2})|p\vec{\omega}\right)}{\prod\limits_{(n_1,n_2,n_3)\in s_{q_1+q_2}(\eta)}\Gamma_3\left(\omega_1+\omega_2+\sum_{i=1}^3\omega_in_i|p\vec{\omega}\right)}\\
&\times\frac{\prod\limits_{(n_1,n_2,n_3)\in s_0(\eta)}\Gamma_3\left(-\mu+\sum_{i=1}^3\omega_i(n_i+\frac{1}{2})|p\vec{\omega}\right)\prod\limits_{(n_1,n_2,n_3)\in s_0(\eta)}\Gamma_3\left(\sum_{i=1}^3\omega_i(n_i+1)|p\vec{\omega}\right)}{\prod\limits_{(n_1,n_2,n_3)\in s_{q_2}(\eta)}\Gamma_3\left(\omega_1+\omega_3+\sum_{i=1}^3\omega_in_i|p\vec{\omega}\right)\prod\limits_{(n_1,n_2,n_3)\in s_{q_1}(\eta)}\Gamma_3\left(\omega_2+\omega_3+\sum_{i=1}^3\omega_in_i|p\vec{\omega}\right)}\,,
\end{aligned}
\end{equation}
where $\Gamma_3(z|\vec{\omega})$ is the Barnes triple Gamma function, defined in \eqref{eqn:multigamma} and $p\vec{\omega}=(p\omega_1,p\omega_2,p\omega_3)$.

\subsection{ Unrefined limits}
\subsubsection{Chiral algebra limit}
The chiral algebra limit is defined by
\begin{equation}
\mu\to\frac{1}{2}(\omega_1+\omega_2-\omega_3)=\omega_1+\omega_2-\frac{3}{2}=\frac{1}{2}+a_1+a_2\,.
\end{equation} 
This is equivalent to $\pbf^2\to\sqrt{\frac{\qbf_1\qbf_2}{\qbf_3}}$. Note that this distinguishes a particular fugacity, however, by permuting $q_i,a_i\leftrightarrow q_j,a_j$ appropriately one may obtain the other limits. The Lens space index is then
\begin{equation}
\begin{aligned}
\mathcal{I}_{p}\left(\qbf_1,\qbf_2,\qbf_3,\sqrt{\frac{\qbf_1\qbf_2}{\qbf_3}}\right)&=\Tr(-1)^Fe^{-\beta\left(E-R_1\right)-\beta\left(a_1\left(h_1-h_3+\frac{R_2-R_1}{2}\right)+a_2\left(h_2-h_3+\frac{R_2-R_1}{2}\right)\right)}\\
&=\Tr(-1)^Fe^{-\beta\left(E-R_1\right)-\beta\left(a_1\left(h_1-h_3+\frac{R_2-R_1}{2}\right)+a_2\left(h_2-h_3+\frac{R_2-R_1}{2}\right)\right)}
\end{aligned}
\end{equation}

We can also consider the further specialisation to
\begin{equation}
\mu=\frac{1}{2}\,,\quad a_1=a_2=a_3=0\,,
\end{equation}
or, equivalently, $\qbf_i=\pbf^4=e^{-\beta}$. This preserves $8$ supercharges $Q^{\pm\pm}_{---}$, $Q^{\pm\pm}_{+++}$ preserved by the Lens space for $p\geq 2$. The $\frac{1}{2}$-BPS Lens space index is given by
\begin{equation}
\mathcal{I}^{\left(\frac{1}{2}\right)}_{p}\left(e^{-\beta}\right):=\mathcal{I}_{p}\left(e^{-\beta},e^{-\beta},e^{-\beta},e^{-\frac{\beta}{2}}\right)=\Tr(-1)^Fe^{-\beta\left(E-R_1\right)}=\Tr(-1)^F\qbf_3^{2E-2R_1}\,,
\end{equation}
it commutes with four supercharges $Q_{---}^{+\pm}$, $Q_{+++}^{-\mp}$. For example, for $p=3$, $q_1=q_2=1$ we have
\begin{equation}
\mathcal{I}^{\left(\frac{1}{2}\right)}_{p}=\PE\left[\frac{\qbf_3 +\qbf^2_3 - 8 \qbf_3^3 + 7 \qbf_3^4 + 7 \qbf_3^{5} - 
 11 \qbf_3^{6} + \qbf_3^{7} + \qbf_3^{8} + \qbf_3^{9}}{(1-\qbf_3)^3}\right]
\end{equation}

\subsubsection{$\frac{1}{2}$-BPS limit}
The $\frac{1}{2}$-BPS limit of the Lens space index is given by taking
\begin{equation}
\text{$\qbf_i\to0$, $\pbf\to0$ with $\mathsf{x}=\frac{\sqrt{\qbf_1\qbf_2\qbf_3}}{\pbf}$ held fixed}
\end{equation}
or equivalently, $\beta\to\infty, \omega_i\to1$ with $-3\beta+\beta\mu=2\log\mathsf{x}$ fixed. For the $\mathfrak{u}(1)$ theory, in this limit the single letter index is actually independent of $p$ and becomes
\begin{equation}
i_p\to\mathsf{x}\,.
\end{equation}


\subsection{$S^5$ partition function review}
Let us review the case $p=1$. 
The $S^5$ partition function of $\mathcal{N}=1^*$ with gauge algebra $\mathfrak{g}=\mathfrak{u}(N)$ can be expressed as \cite{Kim:2012ava,Lockhart:2012vp,Bullimore:2014upa,Kim:2016usy,Kim:2012qf,Kim:2013nva}
\begin{equation}\label{eqn:S5partition}
Z_{S^5}=\int[d\alpha]e^{\frac{2\pi^2(\vec{\alpha},\vec{\alpha})}{\beta\omega_1\omega_2\omega_3}}\prod_{i=1}^3Z^{(i)}_{\text{nek}}\left(\vec{\alpha},\mu+\frac{3\omega_i}{2},\omega_{i+1},\omega_{i+2},\frac{2\pi}{\omega_i},2\pi\beta\right)\,,
\end{equation}
where the index $i$ is taken modulo $3$ and $(\cdot,\cdot)$ denotes the standard metric on $\mathfrak{t}^*$ where $\mathfrak{t}\subset\mathfrak{g}$ is a Cartan subalgebra. The domain of integration is $\iu\mathbb{R}^N$ and the integration measure is given by
\begin{equation}
[d\alpha]=\frac{\iu^{N}}{N!}\prod_{I=1}^Nd\alpha_I\,.
\end{equation}
The partition function is expressed as three copies of the K-theoretic Nekrasov partition function $Z^{(i)}_{\text{nek}}(\vec{\alpha},m,\epsilon_1,\epsilon_2,r,g^2_{\text{YM}})$ on $S_r^1\times\mathbb{R}^4_{\epsilon_1,\epsilon_2}$ where the radius is $r$. The Nekrasov partition function includes both the perturbative and instanton contributions 
\begin{equation}\label{eqn:Nekdecomp}
Z^{(i)}_{\text{nek}}(\vec{\alpha},m,\epsilon_1,\epsilon_2,r,g^2_{\text{YM}})=Z^{(i)}_{\text{pert}}(\vec{\alpha},m,\epsilon_1,\epsilon_2,r)Z^{(i)}_{\text{inst}}(\vec{\alpha},m,\epsilon_1,\epsilon_2,r,g^2_{\text{YM}}).
\end{equation}
Each of the three factors corresponds to the fact that the Localisation locus $\mathcal{L}=\coprod_{i=1}^3 S^1_{2\pi/\omega_i}$ on $S^5$ factorises into three fixed circles of the $U(1)^3$ action \eqref{eqn:u13action}. 
\begin{table}[h!]
\centering
\begin{tabular}{|c c c c c|} 
 \hline
 & $r$ & $\epsilon_1$ & $\epsilon_2$ & $m$ \\ [0.5ex] 
 \hline\hline
  $Z^{(1)}_{\text{nek}}$ & $\frac{2\pi}{\omega_1}$ & $\omega_2$ & $\omega_3$ &$\mu+\frac{3}{2}\omega_1$ \\ 
  $Z^{(2)}_{\text{nek}}$ & $\frac{2\pi}{\omega_2}$ & $\omega_3$ & $\omega_1$ &$\mu+\frac{3}{2}\omega_2$ \\ 
    $Z^{(3)}_{\text{nek}}$ & $\frac{2\pi}{\omega_3}$ & $\omega_1$ & $\omega_2$ &$\mu+\frac{3}{2}\omega_3$ \\ 
 \hline
\end{tabular}
\caption{Arguments for the Nekrasov partition functions associated to the three localisation circles}
\label{table:localisationparams}
\end{table}
\paragraph{Classical contribution}
For the following section, it will be convenient to expresses the $SL(3,\mathbb{Z})$ form explicitly. This can be achieved by writing the classical piece as \cite{Nieri:2013vba}
\begin{gather}
e^{\frac{2\pi^2(\vec{\alpha},\vec{\alpha})}{\beta\omega_1\omega_2\omega_3}}=\prod_{I>J}e^{\frac{-2\pi\iu}{3!}\left[B_{33}\left(\alpha_I-\alpha_J-\frac{\iu\pi}{2N\beta}+\frac{3}{2}\right)-B_{33}\left(-\frac{\iu\pi}{2N\beta}+\frac{3}{2}\right)\right]}=\prod_{i=1}^3Z^{(i)}_{\text{cl}}(\vec{\alpha},\omega_i,\omega_{i+1},\omega_{i+2})\,,\\
Z^{(i)}_{\text{cl}}(\vec{\alpha},\omega_i,\omega_{i+1},\omega_{i+2})=\prod_{I>J}\frac{\Gamma\left(e^{-\alpha_I+\alpha_J+\frac{\iu\pi}{2\beta N}-\frac{3}{2}};e^{-\omega_{i+1}},e^{-\omega_{i+2}}\right)}{\Gamma\left(e^{\frac{\iu\pi}{2\beta N}-\frac{3}{2}};e^{-\omega_{i+1}},e^{-\omega_{i+2}}\right)}\,.
\end{gather}
Here, $\Gamma(z;p,q)$ denotes the Elliptic gamma function, defined as
\begin{equation}
\Gamma(z;p,q):=\prod_{n,m=0}^{\infty}\frac{1-\frac{pq}{z}p^nq^m}{1-zp^nq^m}=\PE\left[\frac{z-pq/z}{(1-p)(1-q)}\right]\,.
\end{equation}
The benefit of this is that the partition function \eqref{eqn:S5partition} can now be manifestly expressed in the $SL(3,\mathbb{Z})$ covariant way, with
\begin{equation}\label{eqn:S5partitionmanifest}
Z_{S^5}=\int[d\alpha]\prod_{i=1}^3\widetilde{Z}^{(i)}_{\text{nek}}\left(\vec{\alpha},\mu+\frac{3\omega_i}{2},\omega_{i+1},\omega_{i+2},\frac{2\pi}{\omega_i},2\pi\beta\right)\,,\quad
\widetilde{Z}^{(i)}_{\text{nek}}:=Z^{(i)}_{\text{cl}}Z^{(i)}_{\text{nek}}\,.
\end{equation}
\paragraph{Perturbative contribution}
We collect all of the perturbative factors $Z_{\text{pert}}=\prod_{i=1}^3Z^{(i)}_{\text{pert}}$
The perturbative piece factorise into a contribution from the vector multiplet and adjoint hypermultiplet 
\begin{equation}\label{eqn:pertcont}
Z_{\text{pert}}\left(\vec{\alpha},\omega_1,\omega_2,\omega_3,\mu\right)=Z_{\text{vec}}\left(\vec{\alpha},\omega_1,\omega_2,\omega_3\right)Z_{\text{hyp}}\left(\vec{\alpha},\omega_1,\omega_2,\omega_3,\mu\right)\,.
\end{equation}
They are given by \cite{Kim:2012ava,Lockhart:2012vp,Kim:2012qf} 
\begin{align}
&Z_{\text{vec}}\left(\vec{\alpha},\omega_1,\omega_2,\omega_3,\mu\right)=\prod_{I,J=1}^NS_3'\left(\alpha_I-\alpha_J|\vec{\omega}\right)\,,\\ &Z_{\text{hyp}}\left(\vec{\alpha},\omega_1,\omega_2,\omega_3,\mu\right)=\prod_{I,J=1}^N\frac{1}{S_3\left(\alpha_I-\alpha_J+\mu+\frac{3}{2}|\vec{\omega}\right)}\,,
\end{align}
where $S_3(z|\vec{\omega})$ is the triple sine function, defined in \eqref{eqn:multisine}. The prime indicates that when $I=J$ the $n_1=n_2=n_3=0$ term in the infinite product product of the triple sine function should be removed.
\begin{comment}
\begin{equation}\label{eqn:pertcont}
Z_{\text{pert}}\left(\vec{\alpha},m,\epsilon_1,\epsilon_2,r\right)=Z_{\text{vec}}\left(\vec{\alpha},m,\epsilon_1,\epsilon_2,r\right)Z_{\text{hyp}}\left(\vec{\alpha},m,\epsilon_1,\epsilon_2,r\right)\,.
\end{equation}  
As in \cite{Bullimore:2014upa} we take the average over the charge conjugation with respect to \cite{Lockhart:2012vp} and therefore
\begin{align}
&Z_{\text{vec}}\left(\vec{\alpha},m,\epsilon_1,\epsilon_2,r\right)=\prod_{e\in\Delta}\tilde{\Gamma}_3'\left((e,\vec{\alpha})\right)^{-1/2}\tilde{\Gamma}_3\left((e,\vec{\alpha})+ 2\pi/r+2\epsilon_+\right)^{-1/2}\,,\label{eqn:pertvec}\\
&Z_{\text{hyp}}\left(\vec{\alpha},m,\epsilon_1,\epsilon_2,r\right)=\prod_{e\in\Delta}\tilde{\Gamma}_3\left((e,\vec{\alpha})+ m+\epsilon_+\right)^{1/2}\,,\label{eqn:perthyp}
\end{align}
where $\epsilon_{\pm}:=\frac{\epsilon_1\pm\epsilon_2}{2}$, $\Delta$ denotes the weights of the adjoint representation of $\mathfrak{g}$ and $\tilde{\Gamma}_3\left(z\right):=\Gamma_3\left(z|2\pi/r,\epsilon_1,\epsilon_2\right)\Gamma_3\left(1-z|2\pi/r,-\epsilon_1,-\epsilon_2\right)$ where $\Gamma_3$ denotes the triple Barnes Gamma function, defined in \eqref{eqn:multigamma}. The prime indicates regularisation required to deal with the Cartan zero modes $(e,\alpha)=0$ such that $\tilde\Gamma_3(0)=\lim_{z\to0}z\Gamma_3(z)$.
\end{comment}
\paragraph{Instanton contribution}
The instanton contribution may be computed from equivariant integration over the moduli space $\mathcal{M}_{k,N}$ of $k$ $U(N)$ instantons. 
$\mathcal{M}_{k,N}$ carries a torus action $T:=T^3_{\epsilon_1,\epsilon_2,m}\times T(U(N))\acts \mathcal{M}_{k,N}$ where $T(G)$ denotes a maximal torus of $G$. $\epsilon_1$, $\epsilon_2$, $m$ and $\vec{\alpha}\in\mathfrak{t}$ . 
The instanton moduli space $\mathcal{M}_{k,N}$ may be described as a algebraic variety using the ADHM construction \cite{Atiyah:1978ri}. Let $V,W$ be vector spaces of dimension $\dim_{\mathbb{C}}V=k$ and $\dim_{\mathbb{C}}W=N$. Let us introduce linear maps \begin{equation}
B^{(l)}:V\to V\,,\quad P:W\to V\,,\quad Q:V\to W\,.
\end{equation}
for $l=1,2,3,4$. The ADHM equations are
\begin{gather}
\mathcal{E}^{(1)}_{\mathbb{C}}:=[B^{(1)},B^{(2)}]+[{B^{(3)}}^{\dagger},{B^{(4)}}^{\dagger}]+PQ\,,\quad\mathcal{E}^{(2)}_{\mathbb{C}}:=[B^{(1)},B^{(3)}]-[{B^{(2)}}^{\dagger},{B^{(4)}}^{\dagger}]\label{eqn:ADHM1}\\
\quad\mathcal{E}^{(3)}_{\mathbb{C}}:=[B^{(1)},B^{(4)}]+[{B^{(2)}}^{\dagger},{B^{(3)}}^{\dagger}]\,,\quad\mathcal{E}_{\mathbb{R}}:=\sum_{l=1}^4[B^{(l)},{B^{(l)}}^{\dagger}]+PP^{\dagger}-Q^{\dagger}Q\,.\label{eqn:ADHM2}
\end{gather}
The moduli space is given by
\begin{equation}
\mathcal{M}_{k,N}:=\left\{B^{(l)}\,,\,P\,,\,Q\,\middle|\,\mathcal{E}_{\mathbb{C}}^{(i)}=\mathcal{E}_{\mathbb{R}}=0\right\}\slash U(k)\,,
\end{equation}
where the $g\in U(k)$ acts by
\begin{equation}\label{eqn:Ukaction}
\left(B^{(1)},B^{(2)},B^{(3)},B^{(4)},P,Q\right)\mapsto\left(gB^{(1)}g^{-1},gB^{(2)}g^{-1},gB^{(3)}g^{-1},gB^{(4)}g^{-1},gP,Qg^{-1}\right)\,.
\end{equation}
The torus action $T^3_{\epsilon_1,\epsilon_2,m}$ acts on the ADHM data by
\begin{equation}
\left(B^{(1)},B^{(2)},B^{(3)},B^{(4)},P,Q\right)\mapsto(e^{\epsilon_1} B^{(1)},e^{\epsilon_2} B^{(2)},e^{m-\epsilon_+}B^{(3)},e^{-m-\epsilon_+}B^{(4)},P,e^{2\epsilon_+}Q)\,.
\end{equation}
The fixed points of the torus action are labelled by $N$-tuples of Young diagrams $\vec{\mu}$ such that $|\vec{\mu}|=k$. See Appendix \ref{eqn:Youngdiags} for various identities. Choosing bases
\begin{equation}
 W=\vecspan_{\mathbb{C}}\left\{w_I\middle|I=1,2,\dots,N\right\}\,,\quad V=\vecspan_{\mathbb{C}}\left\{v^{(i,j)}_{I}\middle|I=1,2,\dots,N,s=(i,j)\in\mu_I\right\}\,.
\end{equation} 
The torus action $T$ acts by
\begin{equation}
w_I\mapsto e^{\alpha_I}w_I\,,\quad v^{(i,j)}_{I}\mapsto e^{(1-i)\epsilon_1+(1-\epsilon_2)}v^{(i,j)}_{I}\,.
\end{equation}
Then
\begin{equation}\label{eqn:fixedpoint}
B^{(1)}v^{(i,j)}_{I}=v^{(i+1,j)}_{I}\,,\quad B^{(2)}v^{(i,j)}_{I}=v^{(i,j+1)}_{I}\,,\quad Pw_I=v^{(1,1)}_{I}\,,\quad Q=B^{(3)}=B^{(4)}=0\,.
\end{equation}
The character of the tangent space $T\mathcal{M}_{k,N}$ at the fixed point labelled by $\vec{\mu}$ is then
\begin{equation}\label{eqn:ind}
\chi_{\vec{\mu}}\left(T\mathcal{M}_{k,N}\right):=\chi_{\vec{\mu}}^{\text{Vec}}+\chi_{\vec{\mu}}^{\text{Hyp}}\,,
\end{equation}
where,
\begin{align}
\chi_{\vec{\mu}}^{\text{Vec}}=\left(W^*V+e^{2\epsilon_+}V^*W-(1-e^{\epsilon_1})(1-e^{\epsilon_2})V^*V\right)\sum_{t\in\mathbb{Z}}e^{\frac{2\pi t}{r}}\,,\quad\chi_{\vec{\mu}}^{\text{Hyp}}=e^{m-\epsilon_+}\chi_{\vec{\mu}}^{\text{Vec}}\,.
\end{align}
Here we abused notation an identified the vector spaces with their characters:
\begin{equation}
V=\sum_{I=1}^N\sum_{(i,j)\in\mu_I}e^{\alpha_I+(1-i)\epsilon_1+(1-j)\epsilon_2}\,,\quad W=\sum_{I=1}^Ne^{\alpha_I}\,,
\end{equation}
where the conjugation flips the sign of the exponents. We also added the dressing by momentum factors along the $S^1$. Using the identity \eqref{eqn:youngsimp} it can be shown that
\begin{equation}
\chi_{\vec{\mu}}^{\text{Vec}}=\sum_{t\in\mathbb{Z}}e^{\frac{2\pi t}{r}}\sum_{I,J=1}^N\sum_{s\in\mu_J}\left(e^{E_{IJ}(s)}+e^{2\epsilon_+-E_{IJ}(s)}\right)\,,
\end{equation}
where,
\begin{equation}
E_{IJ}(s):=\alpha_I-\alpha_J-(\mu^{\trans}_{J;j}-i)\epsilon_1+(\mu_{I;i}-j+1)\epsilon_2\,.
\end{equation}
The contribution of the fixed point $\vec{\mu}$ to the instanton partition is obtained from the character by
\begin{equation}\label{eqn:conversionrule}
\chi_{\vec{\mu}}\left(T\mathcal{M}_{k,N}\right):=\sum_{i}n_ie^{w_i}\quad \to \quad z_{\vec{\mu}}=\prod_{i}w_i^{-n_i}
\end{equation}
Finally, after applying the infinite product \eqref{eqn:Eulersine} for the sine function, the instanton partition function is given by a weighted sum over all possible $N$-tuples $\vec{\mu}$
\begin{align}
&Z_{\text{inst}}\left(\vec{\alpha},m,\epsilon_1,\epsilon_2,r,g^2_{\text{YM}}\right)=\sum_{\vec{\mu}} q^{|\vec{\mu}|}z_{\vec{\mu}}\left(\vec{\alpha},m,\epsilon_1,\epsilon_2,r\right)\,,\\
&z_{\vec{\mu}}\left(\vec{\alpha},m,\epsilon_1,\epsilon_2,r\right)=\prod_{I,J=1}^N\prod_{s\in\mu_J}\frac{\sin\frac{r\left(E_{IJ}(s)+m-\epsilon_+\right)}{2}\sin\frac{r\left(E_{IJ}(s)-m-\epsilon_+\right)}{2}}{\sin\frac{rE_{IJ}(s)}{2}\sin\frac{r\left(E_{IJ}(s)-2\epsilon_+\right)}{2}}\,,
\end{align}
where $q:=e^{-4\pi^2r/g^2_{\text{YM}}}=e^{-2\pi r/\beta}$ and $2\epsilon_{\pm}=\epsilon_1\pm\epsilon_2$. Note that, $(m,\epsilon_1,\epsilon_2))$ are periodic is $\frac{2\pi}{r}$ shifts.

\subsubsection{Unrefined limits}
We would like to discuss the limit that will allow us to compute the unrefined limits of the 6d index. Let us first examine the perturbative piece. The answer, for $\mu=\frac{1}{2}(\omega_1+\omega_2-\omega_3)$ was given in \cite{} and is given by
\begin{equation}
Z_{\text{pert}}\left(\vec{\alpha},\omega_1,\omega_2,\omega_3,\omega_1+\omega_2-\frac{3}{2}\right)=\prod_{I>J}2\sinh\frac{\alpha_I-\alpha_J}{\omega_1}2\sinh\frac{\alpha_I-\alpha_J}{\omega_2}\,.
\end{equation}
Now for the instanton pieces. In this limit, using the periodicity of the variables, Table \ref{} becomes
\begin{table}[h!]
\centering
\begin{tabular}{|c c c c c|} 
 \hline
 & $r$ & $\epsilon_1$ & $\epsilon_2$ & $m$ \\ [0.5ex] 
 \hline\hline
  $Z^{(1)}_{\text{nek}}$ & $\frac{2\pi}{\omega_1}$ & $\omega_2$ & $\omega_3$ &$2 \omega_1 + \frac{\omega_2- \omega_3}{2}\sim\frac{\omega_2-\omega_3}{2}=\epsilon_-$ \\ 
  $Z^{(2)}_{\text{nek}}$ & $\frac{2\pi}{\omega_2}$ & $\omega_3$ & $\omega_1$ &$2 \omega_2 + \frac{\omega_1- \omega_3}{2}\sim\frac{\omega_1-\omega_3}{2}=-\epsilon_-$ \\ 
    $Z^{(3)}_{\text{nek}}$ & $\frac{2\pi}{\omega_3}$ & $\omega_1$ & $\omega_2$ &$\omega_3 + \frac{\omega_1+ \omega_2}{2}\sim\frac{\omega_1+\omega_2}{2}=\epsilon_+$ \\ 
 \hline
\end{tabular}
\caption{Table to test captions and labels}
\label{table:unrefined}
\end{table}
We can see that, from Table \ref{table:unrefined}, in the 5d description the chiral algebra limits correspond to studying the Nekrasov partition function in the cases $m=\pm\epsilon_-$ and $\epsilon_+=m$. It is known that, in these limits
\begin{equation}
z_{\mu}(\vec{\alpha},m=\pm\epsilon_-,\epsilon_1,\epsilon_2,r,g_{\text{YM}}^2)\equiv0\,,\quad z_{\mu}(\vec{\alpha},m=\pm\epsilon_+,\epsilon_1,\epsilon_2,r,g_{\text{YM}}^2)\equiv1\,.
\end{equation}
So, in the former cases the partition function gets non-zero contributions only from the zero instanton sector. In the latter case, the instanton partition function is simply counting coloured young diagrams with a single fugacity $q$ for the number of boxes:
\begin{equation}
Z_{\text{inst}}\left(\vec{\alpha},m=\epsilon_+,\epsilon_1,\epsilon_2,r,g^2_{\text{YM}}\right)=\sum_{\vec{\mu}} q^{|\vec{\mu}|}=q^{-N/24}(q;q)^{-N}=\eta(q)^{-N}\,.
\end{equation} 
So, in total, in this limit \eqref{eqn:S5partition} becomes
\begin{equation}
Z_{S^5}=\frac{1}{\eta(e^{-\frac{4\pi^2}{\beta\omega_3}})^N}\int[d\alpha]e^{\frac{2\pi^2(\vec{\alpha},\vec{\alpha})}{\beta\omega_1\omega_2\omega_3}}\prod_{I>J}2\sin\frac{\alpha_I-\alpha_J}{\omega_1}2\sin\frac{\alpha_I-\alpha_J}{\omega_2}\,.
\end{equation}
This integral was computed in \cite{} and in total reads
\begin{equation}
Z_{S^5}=e^{-\beta E_{p=1}(\mathfrak{u}(N))}\prod_{I=1}^N\frac{1}{(\qbf_3^I;\qbf_3)}\,.
\end{equation}

\subsection{The $L(q_1,q_2,q_3;p)$ partition function}
Let us now discuss how to obtain the partition function for the 5d $\mathcal{N}=1^*$ theory on the Lens space. We \textit{assume} that the $\mathbb{Z}_p$ projections do not change the Localisation structure. In particular we assume that the Localisation locus is still given by $\mathcal{L}=\coprod_{l=1}^3S^1_{2\pi/\omega_l}$.
Under that assumption the partition function takes the form of three copies of an `orbifolded' Nekrasov partition function $\widetilde{Z}_{\text{nek}}^{\text{orb}}=Z_{\text{cl}}^{\text{orb}}Z_{\text{nek}}^{\text{orb}}$. 
Let us focus on the Nekrasov partition function at the $l^{\text{th}}$ localisation circle $S^1_{2\pi/\omega_l}$. Recall that, before the orbifolding, it is given by $Z_{\text{nek}}\left(\vec{\alpha},m,\epsilon_1,\epsilon_2,r,2\pi\beta\right)$ with 
\begin{equation}\label{eqn:S11Id}
m=\mu+3\omega_l/2\,,\quad \epsilon_1=\omega_{l+1}\,,\quad\epsilon_2=\omega_{l+2}\,,\quad r=2\pi/\omega_l\,,
\end{equation}
where, as before, we take $l=1,2,3$ modulo $3$. We have to project onto states left invariant by the orbifold action \eqref{eqn:Lensaction}. 
Note that the third transformation in \eqref{eqn:trans1} implies that the instanton fugacity transforms. For example $\mathbb{Z}_p:q=e^{\frac{-2\pi r}{\beta}}\mapsto e^{\frac{-2\pi rp}{\beta}\left(p+\iu q_l r\right)^{-1}}$. 

According to the Douglas and Moore prescription \cite{Douglas:1996sw} we should also turn on $\pi_1\left(L_p\right)=\mathbb{Z}_p$ valued holonomies breaking $U(N)\to \prod_{A=1}^pU\left(N_A\right)$ such that $\sum_{A=1}^pN_A=N$. 
Since the $U(N)$ was integrated we then have to sum over all possible holonomies in the partition function computation. 
To each $U(N_A)$ we assign equivariant parameters $\alpha_{A,I}$ with $I=1,2,\dots,N_A$ such that $\vec{\alpha}=(\alpha_{1,1},\dots,\alpha_{1,N_1},\dots,\alpha_{p,1},\dots,\alpha_{p,N_p})$. With the identifications \eqref{eqn:S11Id} the orbifold action may be traded for an action on the equivariant parameters
\begin{gather}
\alpha_{A,I}\mapsto\alpha_{A,I}-\frac{2\pi\iu A}{p}\,,\quad m\mapsto m+\frac{6\pi\iu q_l}{2p}\,,\quad \frac{2\pi}{r}\mapsto  \frac{2\pi}{r}+\frac{2\pi\iu q_l}{p} \label{eqn:trans1}\\
\epsilon_1\mapsto \epsilon_1+\frac{2\pi\iu q_{l+1}}{p}\,,\quad \epsilon_2 \mapsto \epsilon_2+\frac{2\pi\iu q_{l+2}}{p}\label{eqn:trans2}\,.
\end{gather}
The Lens space partition function then takes the form
\begin{equation}
Z_{L(q_1,q_2,q_3;p)}=\int[d\alpha]'\prod_{i=1}^3\widetilde{Z}^{\text{orb}}_{\text{nek}}\left(\vec{\alpha},\mu+3\omega_i/2,\omega_{i+1},\omega_{i+2},2\pi/\omega_i,2\pi\beta\right)\,,
\end{equation}
where the measure is given by
\begin{equation}
[d\alpha]'=\sum_{\{N_1,N_2,\dots,N_p|\sum_{A=1}^pN_A=N\}}\frac{\iu^{N}}{N_1!N_2!\dots N_p!}\prod_{A=1}^p\prod_{I=1}^{N_A}d\alpha_{A,I}\,.
\end{equation}
The very first sum is summing over all holonomies.
\subsection{Perturbative contribution}
Let us detail how to implement the orbifold projections at the level of the perturbative part \eqref{eqn:pertcont} of the partition function. Before the orbifold the perturbative piece is built out of triple sine functions which contain infinite products of the form 
\begin{equation}
\prod_{n_1,n_2,n_3=0}^{\infty}\left(z+\vec{n}\cdot\vec{\omega}\right)\,.
\end{equation}
Under the $\mathbb{Z}_p$ transformations \eqref{eqn:trans1}, \eqref{eqn:trans2} let us say that $z\mapsto z+\frac{2\pi\iu b}{p}$. The fixed points of the transformation acting on the above product are labelled by the integers $(n_1,n_2,n_3)$ subject to the condition $\eta+b=0\mod p$ with $\eta=q_1n_1+q_{2}n_2+q_{3}n_3$. Unsurprisingly, the solution is similar to that of \eqref{eqn:testsumres}. The result of keeping only the fixed points of the transformation yields the infinite product
\begin{equation}
\prod_{r_1,r_2,r_3=0}^{\infty}\prod_{(n_1,n_2,n_3)\in s_b(\eta)}\left(z+\vec{\omega}\cdot(\vec{n}+p\vec{r})\right)
\end{equation}
where $s_b(\eta)\subset\mathbb{N}^3$ is the same set that we defined in \eqref{eqn:orbset}. Therefore the orbifold of the perturbative factors is
\begin{align}
&\begin{aligned}
Z_{\text{vec}}^{\text{orb}}=\prod_{A,B=1}^p\prod_{I=1}^{N_A}\prod_{J=1}^{N_B}\prod_{r_1,r_2,r_3=0}^{\infty}\prod_{(n_1,n_2,n_3)\in s_{B-A}(\eta)}\bigg\{\left(\alpha_{A,I}-\alpha_{B,J}+\vec{\omega}\cdot(\vec{n}+p\vec{r})\right)\qquad&\\
\times\left(\alpha_{A,I}-\alpha_{B,J}+\vec{\omega}\cdot(\vec{n}+p\vec{r}+1)\right)\bigg\}&\,,
\end{aligned}
\label{eqn:pertvecorb}
\\
&\begin{aligned}
Z_{\text{hyp}}^{\text{orb}}=\prod_{A,B=1}^p\prod_{I=1}^{N_A}\prod_{J=1}^{N_B}\prod_{r_1,r_2,r_3=0}^{\infty}\prod_{(n_1,n_2,n_3)\in s_{B-A}(\eta)}\bigg\{\left(\alpha_{A,I}-\alpha_{B,J}+\mu+\frac{3}{2}+\vec{\omega}\cdot(\vec{n}+p\vec{r})\right)^{-1}&\\
\times\left(\alpha_{A,I}-\alpha_{B,J}-\mu+\frac{3}{2}+\vec{\omega}\cdot(\vec{n}+p\vec{r}+1)\right)^{-1}\bigg\}&\,.
\end{aligned}\label{eqn:perthyporb}
\end{align}
As before $Z_{\text{vec}}^{\text{orb}}$ should be understood such that if a term with $n_1=n_2=n_3=r_1=r_2=r_3=0$ appears it should be removed by hand.
\subsection{The ramified instanton partition function}
Let us focus on the instanton part of the Nekrasov partition function at the $l^{\text{th}}$ localisation circle $S^1_{2\pi/\omega_l}$.
Now we will describe how to implement the $\mathbb{Z}_p$ action. We decompose the vector spaces $W$, $V$ (for a fixed momentum mode around the $S^1_{2\pi/\omega_l}$) with respect to their $\mathbb{Z}_p$ grading
\begin{equation}
W=\bigoplus_{A=1}^pW_A\,,\quad V=\bigoplus_{A=1}^pV_A\,,
\end{equation}
of dimension $\dim_{\mathbb{C}}W_A=N_A$ and $\dim_{\mathbb{C}}V_A=k_A$. Moreover, we also take the index $A=1,\dots,p$ modulo $p$.
Under the $\mathbb{Z}_p$ action the ADHM data transforms as
\begin{gather}
B^{(1)}\mapsto\gamma^{q_{l+1}}B^{(1)}\,,\quad B^{(2)}\mapsto\gamma^{q_{l+2}}B^{(2)}\,,\quad B^{(3)}\mapsto\gamma^{2q_l}B^{(3)}\,,\quad B^{(4)}\mapsto\gamma^{-q_l}B^{(4)}\,,\\
P\mapsto P\,,\quad Q\mapsto\gamma^{q_{l+1}+q_{l+2}}Q\,.
\end{gather}
Note we used the condition $\sum_{i=1}^3q_i=0$ of equation \eqref{eqn:qspecialisation} to simplify the action on $B^{(3)}$, $B^{(4)}$. In order to have a non-trivial result, following \cite{Douglas:1996sw}, we also quotient by a $\mathbb{Z}_p\hookrightarrow U(k)$ corresponding to \eqref{eqn:Ukaction} with $g=\diag\left(\gamma\mathbb{I}_{k_1},\gamma^2\mathbb{I}_{k_2}\dots,\right)\in U(k)$. This breaks $U(k)\to\prod_{A=1}^pU(k_A)$ with $k=\sum_{A=1}^pk_A$. The surviving components are 
\begin{equation}
\begin{aligned}
&B^{(1)}_{A}\in\Hom\left(V_A,V_{A+q_{l+1}}\right)\,,\quad B^{(2)}_{A}\in\Hom\left(V_A,V_{A+q_{l+2}}\right)\,,\quad B^{(3)}_{A}\in\Hom\left(V_A,V_{A+2q_{l}}\right)\,,\\
&B^{(4)}_{A}\in\Hom\left(V_A,V_{A-q_{l}}\right)\,,\quad P_{A}\in\Hom\left(V_A,V_{A}\right)\,,\quad Q_{A}\in\Hom\left(V_A,V_{A+q_{l+1}+q_{l+2}}\right)\,.
\end{aligned}
\end{equation}
The ADHM equations $\mathcal{E}^{(n=1,2,3,4)}_{\mathbb{C},A}=\mathcal{E}_{\mathbb{R},A}=0$ are given by performing the projections to \eqref{eqn:ADHM1} and \eqref{eqn:ADHM2}. The ramified instanton moduli space is then given by 
\begin{equation}
\mathcal{M}_{\{k_A\},\{N_A\}}^{\mathbb{Z}_p}=\left\{B^{(n)}_A\,,\,P_A\,,\,Q_A\middle|\mathcal{E}^{(n)}_{\mathbb{C},A}=\mathcal{E}_{\mathbb{R},A}=0\right\}\,.
\end{equation}
The fixed points after the $\mathbb{Z}_p$ quotient are still labelled by $N$-tuples of Young diagrams $\vec{\mu}$ which we now label by $\vec{\mu}=\{\mu_{A,I}\}$.
We choose bases $W_A=\vecspan_{\mathbb{C}}\left\{w_{A,I}\middle|I=1,\dots,N_A\right\}$ and $V_{A+q_{l+1}i+q_{l+2}j}=\vecspan_{\mathbb{C}}\left\{v^{(i,j)}_{A+q_{l+1}i+q_{l+2}j,I}\middle|I=1,\dots,N_{A+q_{l+1}i+q_{l+2}j}, (i,j)\in\mu_{A,I}\right\}$. The torus action acts by
\begin{equation}
w_{A,I}\mapsto e^{\alpha_{A,I}}w_{A,I}\,,\quad v^{(i,j)}_{A+q_{l+1}i+q_{l+2}j,I}\mapsto e^{(1-i)\epsilon_1+(1-\epsilon_2)}v^{(i,j)}_{A+q_{l+1}i+q_{l+2}j,I}\,.
\end{equation}
The fixed point configuration is given by the orbifold projection of \eqref{eqn:fixedpoint}, namely
\begin{equation}\label{eqn:fixedpointorb}
B^{(1)}_{A}v^{(i,j)}_{A,I}=v^{(i+1,j)}_{A+q_{l+1},I}\,,\quad B^{(2)}_{A}v^{(i,j)}_{A,I}=v^{(i,j+1)}_{A+q_{l+2},I}\,,\quad P_Aw_{A,I}=v^{(1,1)}_{A,I}\,,\quad Q_A=B^{(3)}_A=B^{(4)}_A=0\,.
\end{equation}
The dimension of $V_B$ is then given by
\begin{equation}\label{eqn:dimVB}
k_B=k_B(\vec{\mu})=\dim_{\mathbb{C}}V_B=\sum_{A=1}^p \sum_{I=1}^{N_A}\sum_{(i,j)\in y_{A,B}^{(I)}}1\,,
\end{equation}
where $y_{A,B}^{(I)}$ is given by
\begin{equation}
y^{(I)}_{A,B}=\left\{(i,j)|(i,j)\in\mu_{A,I},A+q_{l+1}i+q_{l+2}j=B\bmod p\right\}\,.
\end{equation}
For $q_{l+1}=0$ and $q_{l+2}=1$ equation \eqref{eqn:dimVB} reduces to (2.37) of \cite{Kanno:2011fw}. We demonstrate an explicit example in Figure \ref{fig:ramifiedeg}.
\begin{figure}
\begin{tikzcd}
&v_1^{(4,1)}\\
&v_3^{(3,1)}\arrow[u,"B_3^{(1)}"]\arrow[r,"B_3^{(2)}"]&v_1^{(3,2)}\\
&v_2^{(2,1)}\arrow[u,"B_2^{(1)}"]\arrow[r,"B_2^{(2)}"]&v_3^{(2,2)}\arrow[r,"B_3^{(2)}"]\arrow[u,"B_3^{(1)}"]&v_1^{(2,3)}\\
&v_1^{(1,1)}\arrow[u,"B_1^{(1)}"]\arrow[r,"B_1^{(2)}"]&v_2^{(1,2)}\arrow[u,"B_2^{(1)}"]\arrow[r,"B_2^{(2)}"]&v_3^{(1,3)}\arrow[u,"B_3^{(1)}"]\\
w_1\arrow[ur,"P_1"]
\end{tikzcd}
\begin{tikzcd}
&\\
&\\
&\\
&v_3^{(2,1)}\arrow[r,"B_3^{(2)}"]&v_1^{(2,2)}\\
&v_2^{(1,1)}\arrow[u,"B_2^{(1)}"]\arrow[r,"B_2^{(2)}"]&v_3^{(1,2)}\arrow[u,"B_3^{(1)}"]\\
w_2\arrow[ur,"P_2"]
\end{tikzcd}

\begin{tikzcd}
&v_2^{(3,1)}\\
&v_1^{(2,1)}\arrow[r,"B_1^{(2)}"]\arrow[u,"B_1^{(1)}"]&v_2^{(2,2)}\\
&v_3^{(1,1)}\arrow[u,"B_3^{(1)}"]\arrow[r,"B_3^{(2)}"]&v_1^{(1,2)}\arrow[u,"B_1^{(1)}"]\\
w_3\arrow[ur,"P_3"]
\end{tikzcd}
\caption{\textit{Example with $p=3$, $q_l=q_{l+1}=1$, $q_{l+2}=-2$, $N_1=N_2=N_3=1$, $\mu_{1,1}=\{4,3,2\}$, $\mu_{2,1}=\{2,2\}$ and $\mu_{3,1}=\{3,2\}$. Because $N_A=1$ we drop the $I$ indices, for example $v^{(i,j)}_{A,1}=v^{(i,j)}_A$. $k_1=7$, $k_2=5$ and $k_3=6$; in agreement with \eqref{eqn:dimVB}.}}
\label{fig:ramifiedeg}
\end{figure}
Finally the character of $T\mathcal{M}_{\{k_A\},\{N_A\}}^{\mathbb{Z}_p}$ at the fixed point $\vec{\mu}$ is given by the $\mathbb{Z}_p$ invariant part of \eqref{eqn:ind}, namely
\begin{equation}
\chi_{\vec{\mu}}\left(T\mathcal{M}_{\{k_A\},\{N_A\}}^{\mathbb{Z}_p}\right):=\chi_{\vec{\mu},\mathbb{Z}_p}^{\text{Vec}}+\chi_{\vec{\mu},\mathbb{Z}_p}^{\text{Hyp}}\,,
\end{equation}
with 
\begin{align}
&\begin{aligned}
&\chi_{\vec{\mu},\mathbb{Z}_p}^{\text{Vec}}=\sum_{t\in\mathbb{Z}}e^{\frac{2\pi pt}{r}}\sum_{A=1}^pe^{\frac{2\pi A}{r}}\chi_{A}\,,\quad \chi_{\vec{\mu},\mathbb{Z}_p}^{\text{Hyp}}=\sum_{t\in\mathbb{Z}}e^{\frac{2\pi pt}{r}}\sum_{A=1}^pe^{\frac{2\pi A}{r}}e^{m-\epsilon_+}\chi_{A+2q_l}\,,
\end{aligned}\\
&\begin{aligned}
\chi_{A}:=\sum_{B=1}^p&\left[W^*_{B+q_lA}V_B+e^{2\epsilon_+}V^*_{B+q_lA+q_{l+1}+q_{l+2}}W_B-V_{B+q_lA}^*V_{B}\right.\\
&\left.\quad+e^{\epsilon_1}V_{B+q_lA+q_{l+1}}^*V_{B}+e^{\epsilon_2}V_{B+q_lA+q_{l+2}}^*V_{B}-e^{2\epsilon_+}V_{B+q_lA+q_{l+1}+q_{l+2}}^*V_{B}\right]\,,
\end{aligned}
\end{align}
where we used \eqref{eqn:qspecialisation}. As before conjugation reverses the signs of the exponents. We also abused the notation and identified the vector spaces and their characters
\begin{align}
&V_A=\sum_{C,D=1}^p\sum_{I=1}^{N_{q_{l+1}C+q_{l+2}D-A}}\sum_{(pi-C+1,pj-D+1)\in y_{q_{l+1}C+q_{l+2}D-A,A}^{(I)}}e^{\alpha_{q_{l+1}C+q_{l+2}D-A,I}+(C-pi)\epsilon_1+(D-pj)\epsilon_2}\,,\\
&W_A=\sum_{I=1}^{N_{p-A+1}}e^{\alpha_{p-A+1,I}}\,,
\end{align}
under the orbifold $\mathbb{Z}_p:V_A,W_A\mapsto \gamma^AV_A,\gamma^AW_A$. At this point it is very important to stress that we understand $A,B,C,D$ to be taken modulo $p$ when and only when they are considered as indices used to label quantities for example $\alpha_{A,I}=\alpha_{A+p,I}$.
According to the conversion rule \eqref{eqn:conversionrule} we can compute
\begin{equation}
\chi_{\vec{\mu}}\left(T\mathcal{M}_{\{k_A\},\{N_A\}}^{\mathbb{Z}_p}\right)\to z^{\mathbb{Z}_p}_{\vec{\mu}}\left(\vec{\alpha},m,\epsilon_1,\epsilon_2,r\right)\,.
\end{equation}



\begin{subappendices}
\section{Identities}
In this section we collect various definitions and identities used within this paper
\subsection{Young diagrams}\label{eqn:Youngdiags}
We use Greek letters $\mu,\lambda,\nu$ to denote partitions of natural numbers. We denote the empty partition by $\emptyset$. A non-empty partition is a set of integers $\lambda$
\begin{equation}
\lambda_1\geq\lambda_2\geq\dots\lambda_l\geq\dots\geq\lambda_{\ell(\lambda)}>0\,,
\end{equation}
with $\ell(\lambda)$ the number of parts of $\lambda$. This definition is also extended to include $\lambda_{l >\ell(\lambda)}\equiv0$. $\lambda^{\trans}$ is the transpose.
We denote
\begin{equation}
|\lambda|:=\sum_{i=1}^{\ell(\lambda)}\lambda_l\,,\quad ||\lambda||^2:=\sum_{l=1}^{\ell(\lambda)}\lambda^2_l=\sum_{(i,j)\in\lambda^{\trans}}\lambda_l\,.
\end{equation}
We give a box $s$ in the Young diagram coordinates $s=(i,j)$ such that
\begin{equation}
\lambda=\left\{(i,j)|i=1,\dots,\ell(\lambda);j=1,\dots,\lambda_i\right\}\,.
\end{equation}
We will also be interested in $N$-tuples of Young diagrams
\begin{equation}
\vec{\mu}=\left\{\mu_I\middle|I=1,2,\dots,N\right\}\,.
\end{equation}
We write $\lambda_{I;i}$ to denote the number of boxes in the $i^{\text{th}}$ column of the diagram $\lambda_I$.
We will also make use the identity \cite{Nakajima:2003pg,Hosomichi:2014rqa}
\begin{equation}\label{eqn:youngsimp}
\begin{aligned}
&\sum_{(i,j)\in \mu}e^{i\epsilon_1+j\epsilon_2}+\sum_{(i',j')\in \mu'}e^{(1-i')\epsilon_1+(1-j')\epsilon_2}-(1-e^{\epsilon_1})(1-e^{\epsilon_2})\sum_{(i,j)\in\mu}\sum_{(i',j')\in \mu'}e^{(i-j')\epsilon_1+(j-j')\epsilon_2}\\
&=\sum_{(i,j)\in \mu}e^{-(\mu'^{\trans}_{j}-i)\epsilon_1+(\mu_{i}-j+1)\epsilon_2}+e^{2\epsilon_+}\sum_{(i',j')\in \mu'}e^{(\mu^{\trans}_{j'}-i')\epsilon_1-(\mu'_{i'}-j'+1)\epsilon_2}\,.
\end{aligned}
\end{equation}
\subsection{Special functions}
The Euler infinite product representation for the sine function is
\begin{equation}\label{eqn:Eulersine}
\sin(x)=x\prod_{t=1}^{\infty}\left(1-\frac{x^2}{\pi^2t^2}\right)\,.
\end{equation}
The multiple zeta function is 
\begin{equation}
\zeta_r(z,s|\vec{\omega})=\sum_{n_1,n_2,\dots,n_r=0}^{\infty}(\vec{n}\cdot\vec{\omega}+z)^{-s}\,,
\end{equation}
for $z\in\mathbb{C}$ and $\Re s>r$. Multiple gamma functions are defined as
\begin{equation}
\Gamma_r(z|\vec{\omega})=e^{\left.\frac{\partial}{\partial s}\zeta_r(z,s|\vec{\omega})\middle|_{s=0}\right.}\,.
\end{equation}
A regularised infinite product can be defined 
\begin{equation}\label{eqn:multigamma}
\Gamma_r(z|\vec{\omega})\sim\prod_{n_1,n_2,\dots,n_r=0}^{\infty}(\vec{n}\cdot\vec{\omega}+z)^{-1}
\end{equation}
and the multiple sine function is defined as
\begin{equation}\label{eqn:multisine}
S_r(z|\vec{\omega})=\Gamma_r(z|\vec{\omega})^{-1}\Gamma_r(|\vec{\omega}|-z|\vec{\omega})^{(-1)^r}\sim \prod_{n_1,n_2,\dots,n_r=0}^{\infty}\left(\vec{n}\cdot\vec{\omega}+|\vec{\omega}|-z\right)\left(\vec{n}\cdot\vec{\omega}+z\right)^{(-1)^{r+1}}\,,
\end{equation} 
with $|\vec{\omega}|=\omega_1+\dots+\omega_r$. It has the symmetry property
\begin{equation}
S_r(z|\vec{\omega})=S_r(|\vec{\omega}|-z|\vec{\omega})^{(-1)^{r+1}}\,.
\end{equation}
\section{Ramified instanton partition function}

\begin{equation}
\chi_{\vec{\mu}}\left(T\mathcal{M}_{\{k_A\},\{N_A\}}^{\mathbb{Z}_p}\right):=\chi_{\vec{\mu},\mathbb{Z}_p}^{\text{Vec}}+\chi_{\vec{\mu},\mathbb{Z}_p}^{\text{Hyp}}\,,
\end{equation}
with 
\begin{align}
&\begin{aligned}
&\chi_{\vec{\mu},\mathbb{Z}_p}^{\text{Vec}}=\sum_{t\in\mathbb{Z}}e^{\frac{2\pi pt}{r}}\sum_{A=1}^pe^{\frac{2\pi A}{r}}\chi_{A}\,,\quad \chi_{\vec{\mu},\mathbb{Z}_p}^{\text{Hyp}}=\sum_{t\in\mathbb{Z}}e^{\frac{2\pi pt}{r}}\sum_{A=1}^pe^{\frac{2\pi A}{r}}e^{m-\epsilon_+}\chi_{A+2q_l}\,,
\end{aligned}\\
&\begin{aligned}
\chi_{A}:=\sum_{B=1}^p&\left[W^*_{B+q_lA}V_B+e^{2\epsilon_+}V^*_{B+q_lA+q_{l+1}+q_{l+2}}W_B-V_{B+q_lA}^*V_{B}\right.\\
&\left.\quad+e^{\epsilon_1}V_{B+q_lA+q_{l+1}}^*V_{B}+e^{\epsilon_2}V_{B+q_lA+q_{l+2}}^*V_{B}-e^{2\epsilon_+}V_{B+q_lA+q_{l+1}+q_{l+2}}^*V_{B}\right]\,,
\end{aligned}
\end{align}

\begin{align}
&V_A=\sum_{C,D=1}^p\sum_{I=1}^{N_{q_{l+1}C+q_{l+2}D-A}}\sum_{(pi-C+1,pj-D+1)\in y_{q_{l+1}C+q_{l+2}D-A,A}^{(I)}}e^{\alpha_{q_{l+1}C+q_{l+2}D-A,I}+(C-pi)\epsilon_1+(D-pj)\epsilon_2}\,,\\
&W_A=\sum_{I=1}^{N_{p-A+1}}e^{\alpha_{p-A+1,I}}\,,
\end{align}


\begin{equation}
\begin{aligned}
z^{\text{vec}}_1=&\prod_{A,B,C,D=1}^p\prod_{I=1}^{N_{p-B-q_lA+1}}\prod_{J=1}^{N_{\sigma}}\prod_{s\in y_{\sigma,B}^{(J)}}\\
&\times\sin\frac{r}{2p}\left[\alpha_{\sigma-B,J}-\alpha_{p-B-q_lA+1,I}+(C-pi)\epsilon_1+(D-pj)\epsilon_2+\frac{2\pi}{r}A\right]^{-1}
\end{aligned}
\end{equation}
\begin{equation}
\begin{aligned}
z^{\text{vec}}_2=&\prod_{A,B,C,D=1}^p\prod_{I=1}^{N_{p-B+1}}\prod_{J=1}^{N_{\sigma-q_lA-q_{l+1}-q_{l+2}}}\prod_{s\in y_{\sigma-q_lA-q_{l+1}-q_{l+2},B+q_lA+q_{l+1}+q_{l+2}}^{(I)}}\\
&\times\sin\frac{r}{2p}\left[\alpha_{p-B+1,J}-\alpha_{\sigma-q_lA-q_{l+1}-q_{l+2},I}+(C-pi-1)\epsilon_1+(D-pj-1)\epsilon_2+\frac{2\pi}{r}A\right]^{-1}
\end{aligned}
\end{equation}
\begin{equation}
\begin{aligned}
z^{\text{vec}}_3=&\prod_{A,B,C,D,C',D'=1}^p\prod_{I=1}^{N_{\sigma-q_lA}}\prod_{J=1}^{N_{\sigma}}\prod_{s\in y_{\sigma-q_lA,B+q_lA}^{(I)}}\prod_{s'\in y_{\sigma,B}^{(I)}}\\
&\times\sin\frac{r}{2p}\left[\alpha_{\sigma,J}-\alpha_{\sigma-q_lA,I}+(C'-C-pi'+pi)\epsilon_1+(D'-D-pj'+pj)\epsilon_2+\frac{2\pi}{r}A\right]
\end{aligned}
\end{equation}
we defined $\sigma=q_{l+1}C+q_{l+2}D-B$, $\sigma'=q_{l+1}C'+q_{l+2}D'-B$, $s=(C-pi+1,pj-D+1)$ and $s'=(C'-pi'+1,pj'-D'+1)$
\end{subappendices}

\end{document}
