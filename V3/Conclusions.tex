\documentclass[main.tex]{subfiles}
\begin{document}
Over the past decades there has been huge interest in the study of supersymmetric quantum field theories and in particular the computation of exact results. This study is primarily carried out with the hope that a understanding supersymmetric theories will ultimately lead us towards a better understanding of non-supersymmetric theories, such as the standard model of particle physics. This is also of great interest from a mathematical point of view because such study may also lead us towards a mathematically rigorous definition of quantum field theories, similar to that achieved for conformal field theories in two dimensions.

Much of the recent study has been dedicated to theories with $\mathcal{N}=2$ supersymmetry in four dimensions. In this thesis we have initiated a programme towards the study and computation of exact results for certain four dimensional $\mathcal{N}=1$ theories, namely those said to lie within class $\mathcal{S}_k$. They can be described in terms of a twisted compactification of the 6d $(1,0)_{A_{k-1}}$ SCFT on a Riemann surface $\mathcal{C}$. Within this class lies theories which are of similar type to that of $\mathcal{N}=1$ SQCD; namely a theory of $\mathcal{N}=1$ vector multiplets coupled to chiral matter in fundamental representations. $\mathcal{N}=1$ SQCD displays many of the same interesting phenomena as non-supersymmetric QCD, such as confinement. However, class $\mathcal{S}_k$ presents a framework for exploiting new dualities, due to the string and M-theory constructions. Additionally they also inherit many similar properties from their class $\mathcal{S}$ mother theories. Therefore, class $\mathcal{S}_k$ seems like an ideal starting point to begin to compute exact results for $\mathcal{N}=1$ theories, with the hope of being able to understand them as well as we understand $\mathcal{N}=2$ theories.

In this thesis we have, firstly, demonstrated that $\mathcal{N}=1$ analogues of Higgs and Coulomb branches may be defined for theories of class $\mathcal{S}_k$. We have computed the Hilbert series for them and described many of their properties. We have also defined and discussed many interesting limits of the superconformal index. Following this we have described the said Coulomb branch geometry by deriving explicit forms for the Seiberg and Intrilligator curves for the class $\mathcal{S}_k$ theories corresponding to the Riemann surface $\mathcal{C}$ being spheres with two maximal and two minimal punctures. We have matched them to M-theory predictions. Moreover, they can be placed in `Gaiotto form', embedded as a surface within $T^*\mathcal{C}$.

The second accomplishment of this thesis is the computation of the partition function of instantons for a certian subset of class $\mathcal{S}_k$ theories. Our work relied on the correspondence between instantons and D$(-1)$-branes from which we derived the ADHM construction for these theories as a matrix model. This partition function is then identified with conformal blocks of the $\mathcal{W}_{kN}$ algebra \cite{Mitev:2017jqj}, pointing towards the possible existence of further 2d/4d relations for class $\mathcal{S}_k$ theories. It would be very interesting to understand this starting from the 6d $(1,0)_{\Gamma}$ SCFTs and to try to derive this from there, as in \cite{Cordova:2016cmu}. Along this direction we have investigated some $\frac{1}{2}$-BPS defects in six dimensional $\mathcal{N}=(1,0)$ SCFTs. We have computed the elliptic genus of the 2d theory living on the world-volume of the BPS strings of the 6d theories in the presence of the defect. The strings provide the main contribution to the $T^2\times\mathbb{R}^4$ BPS-partition function for the 6d theory and it can be written in a expansion over Elliptic genera.

We have also studied $\mathcal{N}=3$ theories, obtained via discrete gauging of $\mathcal{N}=4$ SYM. The discrete group that we gauge lies within the $SL(2,\mathbb{Z})$ S-duality group, of which the subgroup enhances to a symmetry at certian points of the conformal manifold, aswell as a factor within the $SU(4)$ R-symmetry group. In particular we focused on the structure of their moduli spaces and computed its Hilbert series. In certain special cases we are also able to compute the fully refined supersymmetric index. It would be worth trying to extend this computation to the fully refined superconformal index, other limits of the superconformal index, or even other types of partition function, e.g. the $\mathbb{S}^4$ partition function for both the S-fold theories and discrete gaugings of $\mathcal{N}=4$ SYM.

\end{document}
