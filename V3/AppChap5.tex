\documentclass[main.tex]{subfiles}
\begin{document}
\section{Preserved Superconformal Algebra}\label{sec:preservedSCA}
\subsection{Even Subalgebra}
The even subalgebra of $\mathfrak{psu}(2,2|4)$ is $\mathfrak{b}=\mathfrak{so}(5,1)\oplus\mathfrak{su}(4)$ which we take to be generated by $M^{\mu\nu},K_{\mu},P^{\mu},E$ with $\mu,\nu=1,2,3,4$ and $R_I^J$, $I,J=1,2,3,4$. The Cartans of $\mathfrak{su}(4)$ are $R_i=R_i^i-R^{i+1}_{i+1}$ with $i=1,2,3$. We wish to discuss which generators are preserved by the S-folding/discrete gauging procedure.
Recall that $SL(2,\mathbb{Z})$ transformations can be defined such that they commute with the generators of $\mathfrak{b}$ \cite{Kapustin:2006pk}. In particular $[s_k,\mathfrak{b}]=0$. Hence $s_k$ acts non-trivially only on the fermionic subalgebra which we will discuss momentarily. Hence the subalegbra of $\mathfrak{b}$ preserved by the S-folding/discrete gauging is simply the centraliser of $r_k=\frac{R_1}{2}+R_2+\frac{3R_3}{2}=\frac{1}{2}\sum_{i=1}^3R_i^i-\frac{3}{2}R_4^4$ modulo $k$ in $\mathfrak{b}$. Clearly $\left[r_k,\mathfrak{so}(5,1)\right]=0$. On the other hand, using $\left[R_I^J,R^P_Q\right]=\delta_Q^JR_I^P-\delta^P_IR_Q^J$ it can be shown that
\begin{equation}
[r_k,R^J_I]=\begin{cases}
0&I,J\in\{1,2,3\}\,,\\
0&I=J=4\,,\\
2R^4_I&I\in\{1,2,3\}\,,J=4\,,\\
-2R_4^J&I=4\,,J\in\{1,2,3\}\,.
\end{cases}
\end{equation}
Therefore, the subalgebra of $\mathfrak{su}(4)$ preserved by $r_{k\geq3}$ are given by the $R_I^J$ with $I,J=1,2,3$ and $R_4^4$. These generators span a $\mathfrak{su}(3)\oplus\mathfrak{u}(1)$ algebra. Note however that, since we quotient by $e^{\frac{2\pi\iu}{k}r_k\newdot s_k}$, when $k=1,2$ the full $\mathfrak{su}(4)$ is preserved.

\subsection{Odd Subalgebra} 
The odd subalgebra of $\mathfrak{psu}(2,2|4)$ is spanned by nilpotent generators (supercharges) which sit in representations of the bosonic subalgebra $\mathfrak{b}$. Any representation of $\mathfrak{b}$ can be decomposed into representations of a maximal compact subalgebra $\mathfrak{u}(1)_E\oplus\mathfrak{su}(2)_1\oplus \mathfrak{su}(2)_2\oplus \mathfrak{su}(4)$. The supercharges are then given by
\begin{gather}
\Qfour_{\alpha}^I\in\left(\frac{1}{2},\mathbf{2},\mathbf{1},\mathbf{4}\right)\,,\quad \widetilde{\Qfour}_{\dot\alpha I} \in\left(\frac{1}{2},\mathbf{1},\mathbf{2},\mathbf{\overline{4}}\right)\,,\\ \Sfour^{\alpha}_I\in\left(-\frac{1}{2},\mathbf{\overline{2}},\mathbf{1},\mathbf{\overline{4}}\right)\,,\quad \widetilde{\Sfour}^{\dot\alpha I}\in\left(-\frac{1}{2},\mathbf{1},\mathbf{\overline{2}},\mathbf{4}\right)\,.
\end{gather}
The action on the supercharges is then given by
\begin{gather}
[r_k,\Qfour^{I}_{\alpha}] = \begin{cases}
\Qfour^{I}_{\alpha} & I=1,2,3\\
-3\Qfour^{4}_{\alpha} &I=4
\end{cases}\,,\quad[r_k,\widetilde{\Qfour}_{\dot\alpha I}]=\begin{cases}
-\widetilde{\Qfour}_{\dot\alpha I} & I=1,2,3\\
3\widetilde{\Qfour}_{\dot\alpha 4} &I=4
\end{cases} \, ,\\
[r_k,\Sfour_{I}^{\alpha}]=\begin{cases}
-\Sfour^{I}_{\alpha} & I=1,2,3\\
3\Sfour^{4}_{\alpha} &I=4
\end{cases}\,,\quad[r_k,\widetilde{\Sfour}^{\dot\alpha I}] = \begin{cases}
\widetilde{\Sfour}_{\dot\alpha I} & I=1,2,3\\
-3\widetilde{\Sfour}_{\dot\alpha 4} &I=4
\end{cases} \, ,
\end{gather}
On the other hand, $s_k$ acts on the supercharges by \cite{Kapustin:2006pk,Garcia-Etxebarria:2015wns,Garcia-Etxebarria:2016erx}
\begin{gather}
[s_k,\Qfour^{I}_{\alpha}]=-\Qfour^{I}_{\alpha} \,,\quad  [s_k,\widetilde{\Qfour}_{\dot\alpha I}]=\widetilde{\Qfour}_{\dot\alpha I} \, ,\\ [s_k,\Sfour_{I}^{\alpha}]=\Sfour_{I}^{\alpha} \,,\quad  [r_k,\widetilde{\Sfour}^{\dot\alpha I} ]=-\widetilde{\Sfour}_{\dot\alpha I} \,.
\end{gather}
Therefore, for $k\geq 3$, quotienting by $e^{\frac{2\pi\iu}{k}(r_k\newdot s_k)}\in\mathbb{Z}_k$ preserves 12 Poincar\'e supercharges and 12 conformal supercharges giving rise to $\mathcal{N}=3$ superconformal symmetry in four dimensions.
All in all, for $k\geq3$, a full $\mathfrak{su}(2,2|3)\subset\mathfrak{psu}(2,2|4)$ superconformal algebra is preserved.

\section{Indices of \texorpdfstring{$\mathfrak{su}(2,2|2)$}{su(2,2|2)} Multiplets}\label{sec:indicies}
\renewcommand{\arraystretch}{1.2}
Long multiplets $\mathcal{A}^{E}_{R,r,(j_1,j_2)}$ are generic, unitary, modules of the $\mathfrak{su}(2,2|2)$ superconformal algebra. The multiplets are labelled by the values of the highest weight state (superconformal primary) $(E,R,r,j_1,j_2)$ under the maximal bosonic subalgebra \eqref{eqn:su222maximal}. When the some of representation labels take on certain values the superconformal primary is annihilated by (linear combinations of) some of the supercharges $\Qfour_{\alpha}^I$, $\widetilde{\Qfour}_{\dot{\alpha} I}$ and the multiplet is said to be shortened. The superconformal index \eqref{eqn:SCI} counts short multiplets modulo those that can recombine into long multiplets. The recombination rules are given by \cite{DolanOsborn}
\begin{equation}
\mathcal{A}_{R,r,(j_1,j_2)}^{2R+r+2j_1+2}\iso \mathcal{C}_{R,r,(j_1,j_2)}\oplus\mathcal{C}_{R+\frac{1}{2},r+\frac{1}{2},\left(j_1-\frac{1}{2},j_2\right)}\label{eqn:short1} \, , 
\end{equation}
\begin{equation}
\mathcal{A}_{R,r,(j_1,j_2)}^{2R-r+2j_2+2}\iso \overline{\mathcal{C}}_{R,r,(j_1,j_2)}\oplus\overline{\mathcal{C}}_{R+\frac{1}{2},r-\frac{1}{2},\left(j_1,j_2-\frac{1}{2}\right)} \, , 
\end{equation}
\begin{equation}
\begin{aligned}\mathcal{A}_{R,j_1-j_2,(j_1,j_2)}^{2R+j_1+j_2+2}\iso& \hat{\mathcal{C}}_{R,(j_1,j_2)}\oplus\hat{\mathcal{C}}_{R+\frac{1}{2},\left(j_1-\frac{1}{2},j_2\right)}\oplus\hat{\mathcal{C}}_{R+\frac{1}{2},\left(j_1,j_2-\frac{1}{2}\right)}\\
&\oplus\hat{\mathcal{C}}_{R+1,\left(j_1-\frac{1}{2},j_2-\frac{1}{2}\right)}\end{aligned}\,.
\end{equation}
By formally allowing the $j_1,j_2$ to take on the value $-\frac{1}{2}$ we can write
\begin{equation}
\mathcal{C}_{R,r,\left(-\frac{1}{2},j_2\right)}\iso\mathcal{B}_{R+\frac{1}{2},r+\frac{1}{2},(0,j_2)}\,,\quad\overline{\mathcal{C}}_{R,r,\left(j_1,-\frac{1}{2}\right)}\iso\overline{\mathcal{B}}_{R+\frac{1}{2},r-\frac{1}{2},(j_1,0)}\,,\label{eqn:Semishort}
\end{equation}
\begin{equation}
\hat{\mathcal{C}}_{R,\left(-\frac{1}{2},j_2\right)}\iso\mathcal{D}_{R+\frac{1}{2},(0,j_2)}\,,\quad  \hat{\mathcal{C}}_{R,\left(j_1,-\frac{1}{2}\right)}\iso\overline{\mathcal{D}}_{R+\frac{1}{2},(j_1,0)}\,,
\end{equation}
\begin{equation}
\hat{\mathcal{C}}_{R,\left(-\frac{1}{2},-\frac{1}{2}\right)}\iso\mathcal{D}_{R+\frac{1}{2},\left(0,-\frac{1}{2}\right)}\iso\, \overline{\mathcal{D}}_{R+\frac{1}{2},\left(-\frac{1}{2},0\right)}\iso\hat{\mathcal{B}}_{R+1}\label{eqn:short2}\,,
\end{equation}
for $R\geq0$. Equations \eqref{eqn:short1}-\eqref{eqn:short2} constitute the most general recombination rules for any unitary $\mathcal{N}=2$ SCFT.
\begin{table}
\centering
\resizebox{\columnwidth}{!}{%
\begin{tabular}{|c|c|c|c|c|}
\hline
\multicolumn{4}{|c|}{Shortening Conditions} & Multiplet \\
\hline
\hline
$\mathcal{B}_1$ & $\Qfour_{1\alpha}|R,r \rangle^{h.w.}=0$ & $j_1=0$ & $E=2R+r$ & $\mathcal{B}_{R,r(0,j_2)}$\\
\hline
$\overline{\mathcal{B}}_2$ & $\widetilde{\Qfour}_{2\dot{\alpha}}|R,r \rangle^{h.w}=0$ & $j_2=0$ & $E=2R-r$ & $\overline{\mathcal{B}}_{R,r(j_1,0)}$\\
\hline
$\mathcal{E}$ & $\mathcal{B}_1 \cap \mathcal{B}_2 $ & $R=0$ & $E=r$ & $\mathcal{E}_{r(0,j_2)}$\\
\hline
$\mathcal{\overline{E}}$ & $\overline{\mathcal{B}}_1 \cap \overline{\mathcal{B}}_2$ & $R=0$ & $E=-r$ & $\mathcal{\overline{E}}_{r(j_1,0)}$\\
\hline
$\hat{\mathcal{B}}$ & $\mathcal{B}_1 \cap \overline{\mathcal{B}}_2$ & $r=0, j_1,j_2=0$ & $E=2R$ & $\hat{\mathcal{B}}_{R}$\\
\hline
\hline
$\mathcal{C}_1$ & $\epsilon^{\alpha\beta}\Qfour_{1\beta}|R,r\rangle_{\alpha}^{h.w.} = 0 $ &  & $E=2+2j_1+2R+r$  & $\mathcal{C}_{R,r(j_1,j_2)}$\\
& $(\Qfour_{1})^2|R,r \rangle^{h.w.}=0 \ \textrm{for} \ j_1=0$ & & $E=2+2R+r$ & $\mathcal{C}_{R,r(0,j_2)}$\\
\hline
$\mathcal{\overline{C}}_2$ & $\epsilon^{\dot{\alpha}\dot{\beta}}\widetilde{\mathcal{Q}}_{2\dot{\beta}}|R,r\rangle_{\dot{\alpha}}^{h.w.} = 0 $ &  & $E=2+2j_2+2R-r$  & $\mathcal{\overline{C}}_{R,r(j_1,j_2)}$\\
& $(\mathcal{\widetilde{Q}}_{2})^2|R,r \rangle^{h.w.}=0 \ \textrm{for} \ j_2=0$ & & $E=2+2R-r$ & $\mathcal{\overline{C}}_{R,r(j_1,0)}$\\
\hline
& $\mathcal{C}_1 \cap \mathcal{C}_2$ & $R=0$ & $E=2+2j_1+r$ & $\mathcal{C}_{0,r(j_1,j_2)}$\\
\hline
& $\mathcal{\overline{C}}_1 \cap \mathcal{\overline{C}}_2$ & $R=0$ & $E=2+2j_2-r$ & $\mathcal{\overline{C}}_{0,r(j_1,j_2)}$\\
\hline
$\mathcal{\hat{C}}$ & $\mathcal{C}_1 \cap \overline{\mathcal{C}}_2 $ & $r=j_2 -j_1$ & $E=2+2R+j_1+j_2$  & $\mathcal{\hat{C}}_{R(j_1,j_2)}$\\
\hline
& $\mathcal{C}_1 \cap \mathcal{C}_2 \cap \mathcal{\overline{C}}_1 \cap \mathcal{\overline{C}}_2$ & $R=0,r=j_2-j_1$ & $E=2+j_1+j_2$ & $\mathcal{\hat{C}}_{0(j_1,j_2)}$\\
\hline
\hline
$\mathcal{D}$ & $\mathcal{B}_1 \cap \mathcal{\overline{C}}_2$ & $r=j_2+1$ & $E=1+2R+j_2$ & $\mathcal{D}_{R(0,j_2)}$\\
\hline
$\mathcal{\overline{D}}$ & $\mathcal{\overline{B}}_2 \cap \mathcal{C}_1$ & $-r=j_1+1$ & $E=1+2R+j_1$ & $\mathcal{\overline{D}}_{R(j_1,0)}$\\
\hline
& $\mathcal{E} \cap \mathcal{\overline{C}}_2$ & $r=j_2+1, R=0$ & $E=r=1+j_2$ & $D_{0,(0,j_2)}$\\
\hline
& $\mathcal{\overline{E}} \cap \mathcal{C}_1$ & $-r = j_1+1,R=0$ & $E=-r=1+j_1$ & $\mathcal{\overline{D}}_{0,(j_1,0)}$ \\
\hline
\end{tabular}%
}
\caption{\it Shortening conditions and short multiplets $\mathfrak{su}(2,2|2)$.\label{tab:short}}
\end{table}
We have that
\begin{equation}
\label{eq:MultiIndexE}
\mathcal{I}_{\mathcal{E}_{r,(0,j_2)}}=t^{2r}(pq)^r\frac{1-t(pq)^{-1}\chi_1(y)+t^2(pq)^{-2}}{(-1)^{2j_2}(1-t^3y)(1-t^3y^{-1})}\chi_{2j_2}(y)\quad r\geq2\,,
\end{equation}
\begin{equation}
\mathcal{I}_{\mathcal{D}_{0,(0,j_2)}}=\frac{pqt^2\chi_{2j_2}(y)-t^3\chi_{2j_2+1}(y)-t^5pq\chi_{2j_2-1}(y)+t^6\chi_{2j_2}(y)}{(-1)^{2j_2}(1-t^3y)(1-t^3y^{-1})}\,,
\end{equation}
\begin{equation}
\mathcal{I}_{\overline{\mathcal{D}}_{0,(j_1,0)}}=\frac{t^{4j_1+4}}{(pq)^{j_1+1}}\frac{1-(pq)t^2}{(-1)^{2j_1+1}(1-t^3y)(1-t^3y^{-1})}\,,
\end{equation}
\begin{equation}
\mathcal{I}_{\mathcal{C}_{R,r(j_1,j_2)}}=\frac{t^{4+4R+6j_1+2r}}{(pq)^{R+1-r}}\frac{\left(1-t^2pq\right)\left(t^2pq-t^3\chi_1(y)+\frac{t^4}{pq}\right)}{(-1)^{2j_1+2j_2+1}\left(1-t^3y\right)\left(1-t^3y^{-1}\right)}\chi_{2j_2}(y)\,,
\end{equation}
\begin{equation}
\mathcal{I}_{\hat{\mathcal{C}}_{R(j_1,j_2)}}=\frac{t^{6+4R+4j_1+2j_2}}{(pq)^{R+j_1-j_2}}\frac{\left(1-t^2pq\right)\left(\frac{t}{pq}\chi_{2j_2+1}(y)-\chi_{2j_2}(y)\right)}{(-1)^{2j_1+2j_2}(1-t^3y)(1-t^3y^{-1})}\,,
\end{equation}
\begin{equation}
\mathcal{I}_{\overline{\mathcal{E}}_{r,(j_1,0)}}=\mathcal{I}_{\mathcal{E}_{0,(0,0)}}=\mathcal{I}_{\overline{\mathcal{C}}_{R,r(j_1,j_2)}}=\mathcal{I}_{\mathcal{A}^{E}_{R,r(j_1,j_2)}}=0\label{eqn:index0}\,.
\end{equation}
These may be obtained from \cite{Gadde:2011uv} by conjugation (exchanging $r\to-r$, $j_1\leftrightarrow j_2$) and setting $\tau=t^2(pq)^{-1/2}$, $\sigma=ty(pq)^{1/2}$, $\rho=ty^{-1}(pq)^{1/2}$.
By applying \eqref{eqn:Bnmdecomp}-\eqref{eqn:B0n} in combination with \eqref{eqn:Semishort}-\eqref{eqn:index0} one can compute the contribution to the index of the $\mathfrak{su}(2,2|3)$ multiplets of $\hat{\mathcal{B}}_{[R_1,R_2]}$.

\section{Reduction to Three Dimensions}
Let us define $\qbf=e^{-\beta}$ where $\beta$ is the radii of the $\mathbb{S}^1$ factor. Following \cite{Gadde:2011ia,Dolan:2011rp,Rastelli:2016tbz} let us write
\begin{equation}
t=\qbf^{1/3}\,,\quad y=\qbf^{\eta}\,,\quad p=\qbf^{\rho}\,,\quad q=\qbf^{\gamma}\,.
\end{equation}
We can rewrite the Elliptic Gamma functions as
\begin{equation}
\Gamma\left(\qbf^{\alpha};\qbf^{1+\eta},\qbf^{1-\eta}\right)=\prod_{n,m=0}^{\infty}\frac{\left[-\alpha+2+n(1+\eta)+m(1-\eta)\right]_{\qbf}}{\left[\alpha+n(1+\eta)+m(1-\eta)\right]_{\qbf}}
\end{equation}
where $[n]_{\qbf}=(1-\qbf^n)/(1-\qbf)$ is the $q$-number. The $q$-number satisfies $\lim_{\qbf\to1}[n]_{\qbf}=n$. The $\beta\to0$ limit corresponds to $\qbf\to1$. Therefore
\begin{equation}
\lim_{\qbf\to1}\Gamma\left(\qbf^{\alpha};\qbf^{1+\eta},\qbf^{1-\eta}\right)=\prod_{n,m=0}^{\infty}\frac{-\alpha+2+n(1+\eta)+m(1-\eta)}{\alpha+n(1+\eta)+m(1-\eta)}\,.
\end{equation}
We define $\eta=(1-b^2)/(1+b^2)$. We then have
\begin{equation}\label{eqn:Gammalimit}
\lim_{\qbf\to1}\Gamma\left(\qbf^{\alpha};\qbf^{1+\eta},\qbf^{1-\eta}\right)=s_b\left(\frac{\iu Q}{2}(1-\alpha)\right)\,.
\end{equation}
where $Q=b+b^{-1}$ and $s_b(x)$ is the double sine function.
Let us now discuss the limit applied to the index \eqref{eqn:Zkdis}. We may rewrite \eqref{eqn:Zkdis} as
\begin{equation}
\begin{aligned}
&\mathcal{I}^{\mathfrak{u}(1)}_{\mathbb{Z}_k}(t,y,p,q)=\frac{1}{k}\sum_{l=0}^{k-1}\Bigg\{\Gamma\left(\qbf^{2/3+\rho+\gamma-\frac{2\pi\iu l}{k\beta}};\qbf^{1+\eta},\qbf^{1-\eta}\right)\\
&\times\Gamma\left(\qbf^{2/3+\gamma-\rho-\frac{2\pi\iu l}{k\beta}};\qbf^{1+\eta},\qbf^{1-\eta}\right)\Gamma\left(\qbf^{2/3-2\gamma+\frac{2\pi\iu l}{k\beta}};\qbf^{1+\eta},\qbf^{1-\eta}\right)\\
&\times\prod_{n,m=0}^{\infty}\frac{\left[-\frac{2\pi\iu l}{k\beta}+(n+1)(1+\eta)+m(1-\eta)\right]_{\qbf}}{\left[-\frac{2\pi\iu l}{k\beta}+n(1+\eta)+m(1-\eta)+2\right]_{\qbf}}\Bigg\}\\
&\times\prod_{n,m=0}^{\infty}\frac{\left[-\frac{2\pi\iu l}{k\beta}+n(1+\eta)+(m+1)(1-\eta)\right]_{\qbf}}{\left[\frac{2\pi\iu l}{k\beta}+n(1+\eta)+m(1-\eta)+2\right]_{\qbf}}\Bigg\}\,.
\end{aligned}
\end{equation}
It is useful to consider splitting the sum over $l=0,1,\dots,k-1$ in order
to isolate the $l=0$ term as follows
\begin{equation}
\begin{aligned}
&\mathcal{I}^{\mathfrak{u}(1)}_{\mathbb{Z}_k}(t,y,p,q)=\frac{1}{k}\sum_{l=1}^{k-1}\Bigg\{\Gamma\left(\qbf^{2/3+\gamma-\rho-\frac{2\pi\iu l}{k\beta}};\qbf^{1+\eta},\qbf^{1-\eta}\right)\\
&\times\frac{\Gamma\left(\qbf^{2/3-2\gamma+\frac{2\pi\iu l}{k\beta}};\qbf^{1+\eta},\qbf^{1-\eta}\right)\Gamma\left(\qbf^{2/3+\rho+\gamma-\frac{2\pi\iu l}{k\beta}};\qbf^{1+\eta},\qbf^{1-\eta}\right)}{\prod_{n=0}^{\infty}\left[-\frac{2\pi\iu l}{k\beta}+n(1-\eta)\right]_{\qbf}\left[-\frac{2\pi\iu l}{k\beta}+n(1+\eta)\right]_{\qbf}}\\
&\prod_{n,m=0}^{\infty}\frac{\left[-\frac{2\pi\iu l}{k\beta}+n(1+\eta)+m(1-\eta)\right]_{\qbf}}{\left[-\frac{2\pi\iu l}{k\beta}+n(1+\eta)+m(1-\eta)+2\right]_{\qbf}}\\
&\prod_{n,m=0}^{\infty}\frac{\left[-\frac{2\pi\iu l}{k\beta}+n(1+\eta)+m(1-\eta)\right]_{\qbf}}{\left[\frac{2\pi\iu l}{k\beta}+n(1+\eta)+m(1-\eta)+2\right]_{\qbf}}\Bigg\}\\
&+\frac{1}{k}\frac{\Gamma\left(\qbf^{2/3+\gamma-\rho};\qbf^{1+\eta},\qbf^{1-\eta}\right)\Gamma\left(\qbf^{2/3-2\gamma};\qbf^{1+\eta},\qbf^{1-\eta}\right)}{\Gamma\left(\qbf^{4/3-\rho-\gamma};\qbf^{1+\eta},\qbf^{1-\eta}\right)\prod_{n=0}^{\infty}\left[n(1-\eta)\right]_{\qbf}\left[n(1+\eta)\right]_{\qbf}}\,.
\end{aligned}
\end{equation}
Due the form of the denominator in the second line it is clear that in the $\qbf\to1$ limit the factors with $l\neq0$ vanish. Moreover, when $l\neq0$ no regularisation is required. On the other hand the product from $l=0$ requires regularisation. The usual prescription is simply to drop the overall infinite contribution \cite{Gadde:2011ia}, rendering the limit finite. This regularisation is of course independent of $k$. Moreover, following the presciprition of \cite{Gadde:2011ia}, we identify
\begin{equation}
\eta=\frac{1-b^2}{1+b^2}\,,\quad \gamma=\frac{1}{12}+\frac{\sigma}{\iu Q}\,,\quad\rho=\frac{1}{4}-\frac{\sigma}{\iu Q}\,.
\end{equation} 
Applying \eqref{eqn:Gammalimit} we therefore have that
\begin{equation}
\lim_{\qbf\to1}\mathcal{I}^{\mathfrak{u}(1)}_{\mathbb{Z}_k}(t,y,p,q)=\frac{1}{k}s_b\left(\frac{\iu Q}{4}+\sigma\right)s_b\left(\frac{\iu Q}{4}-\sigma\right)\,.
\end{equation}

\section{Vanishing of \texorpdfstring{$\mathbb{Z}_{\kgg}$}{Zn} Anomalies}\label{Sec:thooftanomaly}
One possible obstruction to the ideas that we have discussed in this paper is the potential that the $\mathbb{Z}_{\kgg}\subset SL(2,\mathbb{Z})$ has 't Hooft-anomaly. Since the symmetry is only emergent at strong coupling checking the $\mathbb{Z}_{\kgg}$-anomalies is a non-trivial.
In \cite{Vafa:1994tf} Vafa and Witten studied the S-duality conjecture in topologically twisted $\mathcal{N}=4$ SYM with gauge group $G$ with $Lie(G)=\mathfrak{g}$ simply laced on a four-manifold $\mathcal{M}$. The partition function for the topologically twisted theory is given by \cite{Vafa:1994tf}
\begin{equation}\label{eqn:twistedpartitionfn}
Z_G(\tau)=|Z(G)|^{b_1(\mathcal{M})-1}\sum_{v\in H^2(\mathcal{M},\pi_1(G))}Z_v(\tau)\,,
\end{equation}
with $Z(G)$ the center of $G$ and
\begin{equation}
Z_v(\tau)=e^{-2\pi\iu\tau s}\sum_{K\in\mathbb{Z}-\frac{1}{2}\langle v,v\rangle}\chi\left(\mathbf{M}_{K,v}\right)e^{2\pi\iu\tau K}\,,\quad \hat{Z}_v(\tau):=\eta(\tau)^{-w}Z_v(\tau)\,,
\end{equation}
where $v=w_2(P)\in H^2(\mathcal{M},\pi_1(G))$ is the second Stiefel-Whitney class of the $G$-bundle $P$ over $\mathcal{M}$, $\langle\cdot,\cdot\rangle$ the intersection form on $H^2(\mathcal{M},\pi_1(G))$ and $\mathbf{M}_{K,v}$ is the moduli space of rank $K$ anti-self-dual instantons on $\mathcal{M}$. Additionally \cite{Vafa:1994tf,Labastida:1999ij,Jinzenji:2002if,Wu:2008bv}
\begin{equation}
s=(\rank\mathfrak{g}+1)\chi(\mathcal{M})/4\,,\quad w=-\chi(\mathcal{M})\,.
\end{equation}
Under modular transformations \eqref{eqn:Sdual} the partition function transforms as
\begin{align}
&\hat{Z}_v(\tau+1)=e^{-\pi\iu\left(2s+w/12+\langle v,v\rangle\right)}\hat{Z}_v(\tau)\,,\\
&\hat{Z}_v\left(\frac{-1}{\tau}\right)=\pm\frac{1}{|Z(G)|^{b_2(\mathcal{M})/2}}\sum_{u\in H^2(\mathcal{M},Z(G))}e^{2\pi\iu\langle v,u\rangle}\hat{Z}_u(\tau)\,.
\end{align}
Let $\mathcal{M}=\mathbb{S}^1\times\mathbb{S}^3$. Its Poincar\'e polynomial is given by $P_{\mathbb{S}^1\times\mathbb{S}^3}(x)=1+x+x^3+x^4$ and therefore $b_2(\mathbb{S}^1\times\mathbb{S}^3)=\chi(\mathbb{S}^1\times\mathbb{S}^3)=0$. By Poincar\'e duality $ H^2(\mathbb{S}^1\times\mathbb{S}^3)\iso H_{2}(\mathbb{S}^1\times\mathbb{S}^3)=\{1\}$ in particular this fixes $v=0$. Therefore, on $\mathcal{M}=\mathbb{S}^1\times\mathbb{S}^3$, the partition function \eqref{eqn:twistedpartitionfn} satisfies 
\begin{equation}
Z_G(\tau+1)=Z_G\left(\frac{-1}{\tau}\right)=Z_G(\tau)=Z_{{}^LG}(\tau)\,.
\end{equation}
Since the partition function \eqref{eqn:twistedpartitionfn} is fully $SL(2,\mathbb{Z})$ invariant, following the arguments of \cite{Seiberg:2018ntt}, we can conclude that on $\mathbb{S}^1\times\mathbb{S}^3$ the $\mathbb{Z}_n\subset SL(2,\mathbb{Z})$ symmetries at $\tau$ fixed as in \eqref{eqn:taufix} of (twisted) $\mathcal{N}=4$ SYM have vanishing 't Hooft anomaly. Therefore we expect that they can be consistently gauged.
\section{S-Fold \texorpdfstring{$\stackrel{?}{\equiv}$}{?=} Discrete Gauging}\label{sec:Sfolddiscgauge}
In a few cases some of the theories that can be obtained from $\mathbb{Z}_n$ discrete gauging of $\mathcal{N}=4$ SYM are equivalent to some of the theories $S^N_{k,\ell,p'}$. However, as we will now show, in most cases this is a possibility only when the parent S-fold theory $S^N_{k,\ell}$ has enhanced $\mathcal{N}=4$ supersymmetry. The strategy is simply to compute the possibilities which allow for \eqref{eqn:Sfoldac} to be equal to \eqref{eqn:discgaugeac}. A more refined strategy, via the comparison of the $\frac{1}{8}$-BPS partition functions has been employed in \cite{Arai:2018utu}. 

In the following we limit the discussion to only the connected part of the gauge group of the parent theories.
\paragraph{$\mathfrak{g}=\mathfrak{u}(N)$} Equating \eqref{eqn:discgaugeac} with \eqref{eqn:Sfoldac} we have
\begin{equation}
N^2=kN^2+(2\ell-k-1)N \, .
\end{equation}
This has solution only for $k=\ell=1$ with $N$ arbitrary ($S^N_{1,1}$) and $N=\ell=1$ with $k$ arbitrary ($S^1_{k,1}$). However, in both cases, the S-fold parent theory has an operator of dimension $1$ and therefore supersymmetry is automatically enhanced to $\mathcal{N}=4$ \cite{Aharony:2016kai,Aharony:2015oyb}. Therefore, performing a $\mathbb{Z}_{p'}$ gauging to the S-fold parent is automatically equivalent to making a $\mathbb{Z}_{\kgg}=\mathbb{Z}_{p'}$ discrete gauging to $\mathcal{N}=4$ SYM with gauge algebra $\mathfrak{u}(1)$ since they are the same theory! In both cases, after the discrete gauging, we have the theory $S^1_{k,1,p'}$.
\paragraph{$\mathfrak{g}=\mathfrak{su}(N+1)=A_N$} Equating \eqref{eqn:discgaugeac} with \eqref{eqn:Sfoldac} we have
\begin{equation}
N^2+2N=kN^2+(2\ell-k-1)N\,.
\end{equation}
The only solutions are $N=1$ $\ell=2$ with $k$ arbitrary, $N=2$ $k=3$ $\ell=1$ and $N=3$ $k=2$ $\ell=1$. The parent S-fold theory $S^1_{k,2}$ does not exist as an S-fold for $k\neq2$ \cite{Aharony:2016kai}. $S^2_{3,1}$ and $S^3_{2,1}$ do but they automatically enhance to $\mathcal{N}=4$ supersymmetry and are conjectured to be equivalent to the $\mathfrak{g}=\mathfrak{su}(3)$ and $\mathfrak{g}=\mathfrak{so}(6)$ $\mathcal{N}=4$ theories with gauge coupling $\tau=e^{\pi\iu/3}$ and $\tau=\text{any}$ \cite{Aharony:2016kai}.
\paragraph{$\mathfrak{g}=\mathfrak{so}(2N)=D_N$}Equating \eqref{eqn:discgaugeac} with \eqref{eqn:Sfoldac} gives
\begin{equation}
2N^2-N=kN^2+(2\ell-k-1)N\,.
\end{equation}
Again we have solution for $N=\ell=1$, $k$ arbitrary as well as for $k=2$, $\ell=1$ with $N$ arbitrary. The first case is the same as for $\mathfrak{g}=\mathfrak{u}(1)$. For $k=2$ we always have enhancement to $\mathcal{N}=4$, in our language these theories are $S^N_{2,1}$.
\paragraph{$\mathfrak{g}=\mathfrak{so}(2N+1)=B_N$}Equating \eqref{eqn:discgaugeac} with \eqref{eqn:Sfoldac}
\begin{equation}
2N^2+N=kN^2+(2\ell-k-1)N\,.
\end{equation}
Which has solution only for $k=\ell=2$ with $N$ arbitrary, $N=1$ $\ell=2$ with $k$ arbitrary, $k=4$ $\ell=1$ $N=2$ and $k=3$ $\ell=1$ $N=3$. In the first three cases $S^N_{2,2}$, $S_{k,2}^1$ and $S_{4,1}^2$ always have enhancement to $\mathcal{N}=4$ \cite{Aharony:2016kai}. They have the correct spectrum of Coulomb branch operators to be equivalent to the $\mathfrak{g}=B_N$ or $C_N$, $B_1$ and $B_2\iso C_2$ $\mathcal{N}=4$ theories respectively. In the final case we find that the S-fold $S^3_{3,1}$ has the same central charges as the $\mathcal{N}=4$ $\mathfrak{so}(7)$ theory. Clearly they are not the same theory, however due to the matching of central charges we cannot rule out the possibility that the discrete gauging $S^3_{3,1,p'}$ may yield the same theory as a $\mathbb{Z}_{\kgg}=\mathbb{Z}_{p'}$ gauging of $\mathcal{N}=4$ $\mathfrak{so}(7)$ theory. Our analysis \eqref{eqn:taufix} would seem to imply that this is infact not the case however, since the $S_{3,1}^3$ theory has a discrete $\mathbb{Z}_3$ global symmetry while the $\mathfrak{so}(7)$ theory can only have $\mathbb{Z}_{2},\mathbb{Z}_{4},\mathbb{Z}_{4},\mathbb{Z}_{8}$ discrete symmetry groups.
\paragraph{$\mathfrak{g}=\mathfrak{sp}(N)=C_N$}
Since the degree of the Casimir invariants for $\mathfrak{sp}(N)$ are the same as for $\mathfrak{so}(2N+1)$ the discussion is the same as above.
\paragraph{$\mathfrak{g}=E_6$}
\begin{equation}
78=36k+6(2\ell-k-1)\,.
\end{equation}
There is solution for $k=\ell=2$, the corresponding $S^6_{2,2}$ has a dimension $2$ Coulomb branch operator and therefore $\mathcal{N}=4$ enhancement this S-fold is the standard $\mathfrak{g}=B_6$ or $C_6$ perturbative orientifold.
\paragraph{$\mathfrak{g}=E_7$}
\begin{equation}
133=49k+7(2\ell-k-1)\,,
\end{equation}
There is solution only for $k=3$, $\ell=1$. There is no $\mathcal{N}=4$ enhancement of the corresponding $S_{3,1}^7$ theory.
\paragraph{$\mathfrak{g}=E_8$}
\begin{equation}
248=64k+8(2\ell-k-1)\,,
\end{equation}
There is solution only for $k=4$, $\ell=2$. There is no $\mathcal{N}=4$ enhancement however $S^8_{4,2}$ does not exist as an S-fold \cite{Aharony:2016kai}.
\paragraph{$\mathfrak{g}=F_4$}
\begin{equation}
52=16k+4(2\ell-k-1)\,,
\end{equation}
There is solution only for $k=4$, $\ell=1$. There is no $\mathcal{N}=4$ enhancement.
\paragraph{$\mathfrak{g}=G_2$}
\begin{equation}
14=4k+2(2\ell-k-1)\,,
\end{equation}
There is solution only for $k=6$ $\ell=1$ and $k=4$, $\ell=2$. In the first case $S^2_{6,1}$ exists as an S-fold but there is $\mathcal{N}=4$ enhancement and it is believed to be equal to the $G_2$ $\mathcal{N}=4$ SYM theory with fixed gauge coupling. In the second case the S-folds of type $S_{4,2}^2$ do not fall into the classification of \cite{Aharony:2016kai} and are believed to not exist as S-folds.
\end{document}
